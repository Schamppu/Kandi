\section{Proof-of-Space (PoSp)\label{posp}}
\begin{otherlanguage}{english}
Proof of Space, josta tässä tutkielmassa käytetään lyhennettä PoSp, kehitettiin ratkaisuna Bitcoinin energiankulutukseen \cite{pospchia1,posp2}. PoSp perustuu siihen, että poiketen PoW:sta louhijat todistavat varanneen tietyn määrän tallennustilaa sen sijaan, että niiden tulisi todistaa käyttäneen tietyn verran laskennallista tehokkuutta ongelman ratkaisulle.

PoSp:n perusajatuksena on, että toisin kuin PoW:ssa, lohkon tiivisteeseen liittyvien ongelmien ratkaisut tallennetaan jo ennen ongelman vastaanottamista louhijan tallennustilaan \cite{posp3spacemint}. Kun ongelma tulee ratkaista, louhijat etsivät tallennustilastaan oikeaa ratkaisua, ja löytäessään ratkaisun validoivat lohkoketjuun tehdyn lisäkysen. Tällöin uuteen lohkoon liittyvän ongelman ratkaisemiseen menee hyvin vähän laskentatehoa, mutta ratkaisujen tallentamiseen vaaditaan paljon tallennustilaa ja aikaa. Näin ollen PoSp:n energiankulutus vastaa käytännössä kuluttajalaitteistolla rakennettua pilvipalvelinta.

Ongelman ratkaisua lukuunottamatta PoSp toimii samalla tavalla kuin PoW (ks. kaavio \ref{fig_pow}): louhija, joka löytää ensimmäisenä oikean ratkaisun lähettää sen muille louhijoille validoitavaksi, ja validoinnin jälkeen oikean ratkaisun löytänyt louhija palkitaan kryptovaluutalla \cite{posp3spacemint}. SpaceMint-projektissa on mainittu PoSp:lle kolme etua verrattuna PoW:n:

\begin{enumerate}
\item Ekologisuus. PoSp tarvitsee toimintaansa tallennustilaa ja suorittaa hakuja tallennustilasta, joka on laskennallisesti huomattavasti edullisempaa kuin PoW:ssa prosessorin laskentatehoa hyödyntävä tiivistealgoritmin tuottamien ongelmien ratkaiseminen.
\item Ekonomisuus. Monilla kuluttajatason päätelaitteilla on huomattava määrä käyttämätöntä levytilaa, joka on taloudellisesti katsoen täysin hyödyntämätöntä. PoW:ssa on taloudellisesti järkevää suorittaa louhintaa vain louhintaan erikoistuneella laitteistolla silloin kun kryptovaluutan hinta kattaa sähkön kulutuksen vaatimat kulut.
\item Tasa-arvoisuus. PoW:ssa louhintaa tehdään lähes ainoastaan siihen erikoistuneissa louhintafarmeissa, kun taas yksittäisen henkilön mahdollisuus tehdä taloudellisesti järkevää louhintaa on pieni. PoSp:ssa on teoreettisesti mahdollista, että louhintaa on voidaan tehdä kuluttajatason laitteistolla.
\end{enumerate}

PoSp:ssa on matalasta energiakulutuksestaan huolimatta ongelmia, mikä on osasyynä siihen miksi konsensusmekanismi on marginaalinen verrattuna PoW:n ja PoS:n. Ekologisuuden kannalta suurin ongelma on, että louhijat ostavat uusia korkean kapasiteetin ja matalan energiankulutuksen kovalevyjä \cite{chiaseagate1, chiaseagate2}. Vaikka louhinta on mahdollista kuluttajalaitteiden käyttämättömällä tallennustilalla, on uusien kovalevyjen ostaminen ja niiden käyttö louhintaan taloudellisesti kannattavaa, mikä vaikuttaa kovalevyjen myyntiin ja valmistukseen. PoSp:n vaikutukset näkyvät jo, vaikka konsensusmekanismi on muihin verrattuna verratain uusi: Seagaten kymmenen teratavun kovalevyjen myynti nousi 2021 vuoden huhtikuussa 240\% edellisen vuoden huhtikuuhun verrattuna, kun uusi PoSp-konsensusmekanismilla toimiva Chia-lohkoketju julkaistiin. Euroopan Unionin komission 2015 vuonna tekemän tutkimuksen \cite[p.~60]{manufacturingcarbon1} mukaan kovalevyistä voidaan kierrättää vain 6,3\% siihen käytetyistä raaka-aineista.

Suurin PoSp lohkoketju, Chia, on tällä hetkellään kryptovaluuttojen markkina-arvotilastossa 244. suurin \cite{Coingecko}. Tämä tutkielma esittelee kyseisen lohkoketjun ja sen toimintaa, sekä tuo esille Chian positiiviset puolet ja siihen liittyviä ongelmia.


%% Kannattaako tätä edes esitellä? \subsection{SpaceMint\label{spacemint}}
\begin{otherlanguage}{english}

SpaceMint

\end{otherlanguage}
\subsection{Chia\label{chia}}
\begin{otherlanguage}{english}\sloppy

Chia-lohkoketjua kehittää Chia Network Inc. yritys, jonka perusti Bram Cohen 2017 \cite{chia1}. Bram Cohen on tullut tunnetuksi
BitTorrent-sovelluksesta. Chia-lohkoketjun pääverkko (mainnet) julkaistiin 2021 neljän vuoden testausjakson jälkeen.

Chia käyttää konsensusmekanisminaan Proof of Space and Timea (PoST) \cite{chia1}. Tämä konsensusmekanismi rakennettiin hyvin samanlaiseksi kuin miten Bitcoinin PoW-konsensus-mekanismi toimii, mutta PoW:n energiakulutusongelma ratkaistaan Chiassa käyttämällä Proof-of-Spacea PoW:n sijaan. Chiassa vaaditaan myös vähintään 51\% konsensus, jotta uusi lohko voidaan hyväksyä osaksi lohkoketjua. Chian etuina Bitcoiniin verrattuna on myös Ethereumin kanssa hyvin samanlainen CLVM (Chia Lisp Virtual Machine). CLVM:llä on mahdollista kirjoittaa samaan tapaan kuin Ethereumissa älysopimuksia (smart contracts), millä voidaan tehdä lohkoketjussa toimivia sovelluksia.

Toimintaperiaatteena Chiassa on tämän kappaleen alussa esitellyn kaltaista konsensusmekanismia hyvin vastaava mekanismi \cite{pospchia1}. Käytännössä siis siinä määrin missä Bitcoin vaatii louhijaa todistamaan käyttämänsä laskentatehon, Chiassa vaaditaan louhijaa todistamaan lohkoketjulle allokoitu tallennustila. Chiassa tämä on tehty niin, että louhija tallentaa ratkaisuja tallennustilaansa, ja kun uusi lohko lisätään lohkoketjuun, tulee louhijan löytää tallennetuista ratkaisuista oikea, ja nopeiten oikean ratkaisun löytänyt käyttäjä palkitaan kryptovaluutalla.

Chian energiankulutus on PoW:lla vastaavanlaisiin (Bitcoin, Ethereum) verrattuna verrattain alhainen \cite{chiaenergy}. Koska Chia ei vaadi konsensusmekanismissaan laskentatehoa ja myös kuluttajatason tallennuslaitteiden energiakulutus on alhainen, on Chian arvioitu kuluttavan vuonna 2021 vain 0,307 terawattituntia energiaa. Chian teoreettisesta siirtotapahtumien määrän maksimista sekunnissa ei ole tehty tarkkaa tutkimusta, mutta koska Chian käyttämä PoSp on hyvin samankaltainen PoW:n kanssa voidaan olettaa siirtotapahtumien määrän sekunnissa olevan maksimissaan noin kaksikymmentä per sekunti. Chian kehittäjä Bram Cohen on myös todennut vuoden 2021 maaliskuussa tehdyssä haastattelussa \cite{bram-cohen-interview} siirtotapahtumien maksimimäärän sekunnissa olevan noin kaksikymmentä. Tämä esitetty arvio voidaan vahvistaa sillä, että Chia kohtasi niin kutsutun "dust storm" -hyökkäyksen lokakuussa 2021, jolloin lohkoketju joutui toimimaan maksimikapasiteetillaan ja tällöin saavutettiin hetkellisesti 25 siirtotapahtumaa sekunnissa, mikä on hieman enemmän kuin arvioitu kapasiteetti \cite{chia-tps}. Chian energiankulutus siirtotapahtumaa kohden on arvioitu käyttämällä tätä lukua ja vuosittaista energiankulutusta tarkempien tilastojen puuttuessa. Näiden lukujen perusteella voidaan laskea Chian energiankulutuksen yhtä siirtotapahtumaa kohden olevan 0,4864 kilowattituntia seuraavalla kaavalla:
\begin{mycapequ}[!htbp]
\begin{equation}
\begin{split}
E_{kok} & = 0,307~TWh \\
 & = 307000000~kWh \\
TX_{kok} & = 20~tx/s \\ 
 & = 631138520~tx/a \\
\frac{E_{kok}}{TX_{kok}} & = \frac{307000000~kWh}{631138520~tx/a} \\
 & = 0,4864~kWh/tx
\end{split}
\end{equation}
\caption{2.1: Lasketaan Chian energiankulutus siirtotapahtumaa kohden. \(E_{kok}\) vastaa Chian kokonaisenergiankulutusta vuodessa, kun taas \(TX_{kok}\) vastaa siirtotapahtumien määrää vuodessa, mikäli Chia toimisi sen maksimikapasiteetilla \(20~tx/s\).}
\end{mycapequ}

Energiankulutus ei kuitenkaan ole lohkoketjujen ekologisuutta vertailtaessa ainut olennainen mittari Chian tapauksessa. Chia-lohkoketjun louhijat ovat joulukuussa 2021 allokoineet 34,7 eksbitavua tallennustilaa yhteensä louhinnalle, mikä vastaa noin neljää miljoonaa kymmenen teratavun kovalevyä \cite{chiaspaceusage}. Euroopan unionin 2015 vuoden tutkimuksen mukaan yhden kovalevyn valmistamiseen kuluu kaksi kilowattia energiaa \cite[p~49--50]{manufacturingcarbon1} ja yhden kaupallisessa käytössä toimivan serveritietokoneen käyttöiän on tutkimuksessa laskettu olevan neljä vuotta. Tämän lisäksi kovalevyjen komponenteista kierrätetään 47\%, ja näiden kierrätettyjen komponenttien valmistamiseen käytetyistä materiaaleista ainoastaan 6,3\% voidaan kierrättää. Kierrätetyistä materiaaleista valtaosa on alumiinia, kun taas kriittisisiä raaka-aineita mitä käytetään kovalevyn piirilevyn ja magneettien valmistukseen ei kierrätetä.

Chia on kuitenkin esittänyt, että lohkoketjun louhijat tulisivat käyttämään tallennustilanaan kaupallisessa käytössä toimineiden servereiden päivityksessä poistuneita tallennuslaitteita \cite{chia-mining}. Väitteessä perusteluksi on annettu, että käytettyjen tallennuslaitteiden ostaminen on louhijoille huomattavasti halvempaa. Koska PoSp:ssa ei tallenneta tallennustilaan mitään kriittistä tietoa voidaan tallennuslaitteita käyttää niiden hajoamiseen saakka.

\end{otherlanguage}

\end{otherlanguage}