\begin{subsection}{Cardano\label{cardano}}

Cardanon on kehittänyt Charles Hoskinsonin perustama IOHK vuonna 2015 \cite{cardanowhitepaper, iohk}, ja Cardanon sekä sen ekosysteemin kehitystä valvoo itsenäinen Cardano Foundation. 

Cardano käyttää PoS-protokollanaan viiden eri akateemisen instituution yhteistyössä kehittämää Ouroboros-lohkoketjuprotokollaa \cite{cardanowhitepaper}. Ouroboros on PoS-konsensusmekanismia soveltava lohkoketjuprotokolla \cite{cardano-ouroboros}, joka on kehitetty ratkaisuksi Bitcoinin korkeaan energiakulutukseen.

Cardanon tavoitteena on rakentua akateemisesti vertaisarvioitujen tutkimuksien pohjalta, ja sen Ourobos-protokolla mahdollistaa useiden eri ominaisuuksien implementoinnin modulaarisesti \cite{cardanowhitepaper}. Ourobos-protokolla mahdollistaa joustavan kehitysprosessin, jonka ansiosta Cardano lohkoketjuna on skaalattavampi ja tulevaisuudenkestävämpi. Cardanossa on erilaisia satunnaislukugeneraattoreita, mitkä ovat muissa lohkoketjuissa harvinaisia sekä mahdollisuus sivuketjuille (side-chains).

Cardanon käyttämä PoS-konsensusmekanismi on energiakulutukseltaan verrattuna PoW- konsensusmekanismeihin alhainen, sillä PoS ei tarvitse lainkaan laskennalliseen tehokkuuteen tai allokoidun tallennustilan suuruuteen perustuvaa todentamista \cite{cardanowhitepaper}. Cardanon energiakulutuksen on arvioitu 2021 olevan 6 gigawattituntia \cite{cardanoenergy}, missä yhden siirtoon vaadittu energiankulutus on keskimäärin 0.5479 kilowattituntia. Ouroboroksesta 2017 vuonna julkaistu tutkimus osoittaa, että Cardano kykeneen nykymuodossaan käsittelemään noin 257 siirtotapahtumaa sekunnissa \cite{cardano-tps}, mutta tämän oletetaan skaalautuvan satoihin tuhansiin mikäli Cardano implementoi niin kutsutun Hydra-menetelmän \cite{cardano-hydra}. Hydra-menetelmässä lohkoketjun siirtotapahtumia voidaan suorittaa päälohkoketjun lisäksi niin kutsutuissa layer-2 ratkaisuissa, joista kukin kykenee Hydra-menetelmän mukaan suorittamaan noin tuhat siirtotapahtumaa sekunnissa. Layer-2 ratkaisut ovat rinnakkaisesti toimivia lohkoketjuja, jotka hyödyntävät päälohkoketjun, eli layer-1:n, konsensusmekanismia toiminnassaan esimerkiksi älysopimuksella. Tällöin ne jakavat päälohkoketjun turvallisuuden. Nämä eroavat sivuketjuista (eng. \textit{sidechains}) siinä, että sivuketjut käyttävät omaa konsensusmekanismiaan ja toimivat täysin erillisesti päälohkoketjusta.

\end{subsection}