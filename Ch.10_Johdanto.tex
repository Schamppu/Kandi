\chapter{Johdanto\label{intro}}

Vuonna 2021 Bitcoinin energiankulutukseksi on arvioitu 181.07 terawattituntia \cite{bitcoinenergy}, joka on kasvanut vuoden 2017 energiankulutuksesta (30.2 terawattituntia \cite{bitcoinenergy1}) 599,6\%. Mikäli Bitcoin olisi valtio, se olisi yksinään energiakulutukseltaan maailman 24. suurin valtio, mikä on enemmän kuin esimerkiksi Puola. Energiakulutuksen kasvun arvioidaan vain jatkuvan edelleen, ja näin ollen Bitcoin ja kryptovaluutat yleisesti ovat esittäneet useita kysymyksiä niiden ympäristöystävällisyydestä ja vastuullisuudesta.

Bitcoin käyttää niin kutsuttua Proof-of-Work (PoW) -konsensusmekanismia \cite{satoshibitcoin}, jolla voidaan todentaa lohkoketjuun tehtävät lisäykset käyttämällä tietty määrä laskennallista tehokkuutta ongelman ratkaisuun. Tämä on aiheuttanut Bitcoinin louhinnassa kilpailutilanteen, missä käyttäjät kilpailevat laskennallisesta tehokkuudesta saavuttaakseen suurempia voittoja louhinnassa.

Tämä tutkielma käsittelee kolmen eri konsensusmekanismin energiankulutusta (Proof-of-Work, Proof-of-Stake ja Proof-of-Space) sekä niiden ympäristöystävällisyyttä. Tutkielma esittelee aluksi lohkoketjujen yleistoimintaperiaatteen, konsensusmekanismien toimintaperiaatteet yksinkertaistetusti ja kaksi konsensusmekanismia käyttävää lohkoketjua.

Konsensusmekanismien ja niitä hyödyntävien lohkoketjujen esittelyn jälkeen tutkielma vertailee niiden energiankulutusta ja ympäristöystävällisyyttä. Tämän lisäksi tutkielma vertailee lohkoketjujen välillä olevia muita tärkeitä aspekteja, kuten niiden turvallisuutta ja skaalautuvuutta. Vaikka tutkielman pääpainona on vertailla konsensusmekanismien energiankulutusta ja vastuullisuutta, on konsensusmekanismeja mahdoton vertailla järkevästi ottamatta huomioon niihin liittyviä muita mahdollisia heikkouksia.

