\section{Proof-of-Work (PoW)\label{pow}}
\begin{otherlanguage}{english}
Proof-of-Work, josta tässä tutkielmassa käytetään lyhennettä PoW, on tutkielman kirjoittamisen hetkellä markkinaosuudeltaan suurin konsensusmekanismi. PoW:ta käyttää kaksi suurinta lohkoketjua: Bitcoin ja Ethereum \cite{Coingecko}.

\begin{figure}[!htbp]
\centering
\fbox{\includegraphics[width=0.98\textwidth]{proof-of-work}}
\caption{Esimerkki Proof-of-Work ja Proof-of-Space konsensusmekanismien toimintaperiaatteesta}
\label{fig_pow}
\end{figure}

\vspace{1mm}

PoW vaatii, että sen turvaamiseen osallistuvat käyttäjät, joita usein kutsutaan louhijoiksi (miners), ratkaisevat ongelman varmistaakseen, että lohkoketjuun lisätty lohko on oikea \cite{blockchain1}. Ongelman ratkaisemisen jälkeen oikean ratkaisun löytäneelle käyttäjälle annetaan palkkio (block reward), ja kaavio \ref{fig_pow} esittää miten uusien lohkojen lisääminen tapahtuu lohkoketjuun. Tätä kutsutaan louhinnaksi (mining), ja tämä tutkielma myös käyttää kyseistä termiä kuvatakseen tätä tapahtumaa. PoW lohkoketjujen sisältämää dataa ja toimintaa esitellään tutkielman johdannossa.

Kaavio \ref{fig_pow} esittää, miten uuden lohkon luominen, varmistaminen, ja lisääminen tapahtuu. Tämä kolmeen vaiheeseen (luonti, louhinta, palkitseminen) jaettu prosessi toimii seuraavalla tavalla:

\begin{enumerate}
\item Luodaan siirtotapahtumia datanaan sisältävä lohko. Tutkielman johdannossa esiteltiin, mitä tyypillisesti lohkoketjun yksi lohko sisältää datanaan. Kaavion \ref{fig_pow} ensimmäinen osa esittää uuden luodun lohkon, missä olennaista louhinnan kannalta on sen tiiviste.
\item Louhijat pyrkivät ratkaisemaan lohkon tiivisteeseen liittyvän ongelman. Koska esimerkiksi Bitcoinissa käytetty SHA-256 tiivistealgoritmi luo annetusta merkkijonosta aina saman tiivisteen, tehdään oikean tiivisteen arvaamisesta louhijoille hankalaa luomalla sille vaikeusaste, mikä saavutetaan lisäämällä tiivisteen alkuun jokin määrä nollia \cite{blockchain1}. Vaikeusaste määräytyy louhijoiden määrän (eli käytännössä kokonaislaskentatehokkuuden) mukaan niin, että uusi lohko luodaan noin joka kymmenes minuutti. Tämän jälkeen louhijat pyrkivät luomaan tiivisteen, minkä alussa on yhtä monta nollaa kuin asetetussa vaikeusasteessa. Koska yksi merkkijono tuottaa tiivistelagoritmilla aina saman tiivisteen, joutuvat louhijat lisäämään tiivisteeseen satunnaisen luvun, mitä kutsutaan nonceksi (number once), jolloin riittävällä määrällä yrityksiä jossakin vaiheessa joku louhijoista saa luotua sellaisen tiivisteen, missä on vaikeuasteen vaatima määrä nollia tiivisteen alussa. Tämä prosessi on yksinkertaistettuna esitetty kaavion \ref{fig_pow} toisessa vaiheessa.
\item Oikean ratkaisun (eli käytännössä oikean tiivisteen tuottaneen noncen) löytänyt käyttäjä lähettää vastauksensa lohkoketjulle ja vastaanottaa ratkaisun löytämisestä palkkion. Tyypillisesti palkkiona käyttäjä saa jonkin määrän lohkoketjun käyttämää kryptovaluuttaa itselleen. Kaavion \ref{fig_pow} viimeinen vaihe käsittelee palkkionjakoa.
\end{enumerate}

PoW konsensusmekanismin vahvuutena on sen suuri hajautuneisuus. Lohkoketjuissa tyypillisesti periaatteena on, että mitä hajautuneempi lohkoketju on, sitä turvallisempi se on, sillä laajasti hajautuneisiin lohkoketjuihin on käytännössä mahdotonta kohdistaa hyökkäys \cite{51attack}. Esimerkiksi Bitcoinia vastaan hyökkääminen vaatisi, että hyökkääjä saisi käyttöönsä yli puolet Bitcoinin louhintaan käytetystä laskentatehosta. Tästä käytetään Bitcoinin tapauksessa nimitystä "51\% Attack".


Tutkielma esittelee seuraavaksi miten kuvailtua konsensusmekanismia käytetään kahdessa suurimmista lohkoketjuista, eli Bitcoinissa ja Ethereumissa.

\begin{subsection}{Bitcoin\label{bitcoin}}
% Oikoluettu: X 

Bitcoin kehitettiin vuonna 2009 nimimerkillä "Satoshi Nakamoto" esiintyneen tuntemattoman tahon toimesta ja siitä julkaistiin sen toimintaperiaatetta \cite{bitcoin1, satoshibitcoin} kuvaava tutkimusartikkeli, jota kryptovaluuttakeskusteluissa kutsutaan \textit{white paperiksi}. Bitcoinin tavoitteina fiat-rahaan (perinteinen raha, kuten eurot ja dollarit) verrattuna on, että sitä voi siirtää ilman rajoituksia, sen siirtomaksut ovat alhaisia ja että jokaisella käyttäjillä on pääsy täydelliseen tilikirjaan joka sisältää kaikki tapahtumat. Näiden lisäksi valuuttaa turvaa konsensusmekanismi ja kryptografia. Kryptovaluuttana Bitcoin on myös luottamukseton (eng. \textit{trustless}), eli valuutta ei vaadi toimiakseen mitään kolmatta osapuolta johon käyttäjien täytyisi luottaa vaan se toimii hajautetusti. Tavoitteet ovat muilta osin Bitcoinissa toteutuneet, mutta siirtomaksut ovat kryptovaluutan hinnan noustua nousseet jo kymmeniin euroihin.

Bitcoinissa on vain yhdenlaisia lohkoja, eli siirtolohkoja (eng. \textit{transaction block}). Valuuttaa siirtäessään käyttäjä joutuu maksamaan siirtomaksun (eng. \textit{transaction fee}) ja siirtotapahtuma lisätään seuraavaan lohkoon, jota ei ole vielä lisätty osaksi lohkoketjua. Kun lohko lisätään osaksi lohkoketjua se varmennetaan, eli louhija ratkaisee lohkoon liittyvän pulman ja saa palkintona sen sisältämissä siirtotapahtumissa maksetut siirtomaksut sekä lohkopalkkion (eng. \textit{block reward}). Lohkopalkkioiden määrä puolittuu noin joka neljäs vuosi, ja Bitcoineja tulee olemaan kokonaisuudessaan 21 miljoonaa kappaletta \cite{satoshibitcoin}.

Bitcoinissa ongelman ratkaisemiseksi ja palkkion saamiseksi tärkein tekijä on käyttäjän louhintaan käyttämän tietokoneen laskennallinen tehokkuus \cite{bitcoin1}. Bitcoinissa palkkion saannin todennäköisyys on suoraan verrannollinen käyttäjän laskennallisen tehon suhteeseen lohkoketjun kokonaislaskentatehosta. Tästä syystä valtaosa louhijoista liittyykin niin kutsuttuihin louhintavarantoihin (eng. \textit{mining pool}). Louhintavarannoissa louhijat yhdistävät laskentatehonsa ja saavat louhintavarannon voittamista palkkioista osuutensa riippuen siitä, kuinka suuri osuus heidän laskentatehostaan on louhintavarannon kokonaislaskentatehosta.

Louhinnassa palkkion voittamisen todennäköisyyden riippuessa täysin laskennallisesta tehosta Bitcoinin louhinta on aiheuttanut kilpailua laskentatehosta käyttäjien välillä. Laitteistokilpailusta ja johtavasta asemastaan suurimpana kryptovaluuttana Bitcoin vaatiikin paljon energiaa: vuonna 2021 Bitcoinin louhinnan energiankulutuksen on arvioitu olevan 200,57 terawattituntia \cite{bitcoinenergy}, ja yksi siirtotapahtuma kuluttaa keskimäärin noin 2006,54 kilowattituntia energiaa. Mikäli Bitcoin olisi valtio, olisi Bitcoin energiankulutukseltaan suurempi kuin Thaimaa. Bitcoinin energiankulutus onkin herättänyt keskustelua lohkoketjujen ympäristöystävällisyydestä ja ollut suurena vaikuttajana siihen, miksi ympäristöystävällisempiä uusia konsensusmekanismeja on kehitetty.

Bitcoinin skaalautuvuus on myös saanut osakseen arvostelua. Bitcoin pystyy käsittelemään tällä hetkellä ainoastaan seitsemän siirtotapahtumaa sekunnissa \cite{bitcoin-tps}, ja laskettu teoreettinen maksimi Bitcoinia skaalatessa on 27 siirtotapahtumaa sekunnissa mikäli yhden lohkon koko pysyisi yhdessä megatavussa. Verrattuna siihen, että PoS-konsensusmekanismilla toimivat lohkoketjut pystyvät käsittelemään jo nyt kymmenkertaisia määriä siirtotapahtumia sekunnissa \cite{algorandtech, cardano-ouroboros} on Bitcoinin käyttämän PoW-konsensusmekanismin kohtaama kritiikki tehottomana ymmärrettävää.

\end{subsection}
\begin{subsection}{Ethereum\label{ethereum}}

Ethereumin kehitti kanadalais-ukrainalainen Vitalik Buterin, ja se julkaistiin 2014 \cite{buterin2017ethereum}. Ethereum kehitettiin Bitcoinin perustalle, mutta eroaa Bitcoinista radikaalisti siinä, että Ethereum sisältää Turing-täydellisen Ethereum-virtuaalikoneen (Ethereum Virtual Machine, EVM).

Ethereum rakentuu samalle PoW-konsensusmekanille, mitä Bitcoin myös käyttää \cite{buterin2017ethereum}. Louhinta tapahtuu laskennallisesti vaativan ongelman ratkaisulla, mistä oikean ratkaisun löytänyt käyttäjä palkitaan antamalla korvauksena Ether-virtuaalivaluuttaa. Siirtotapahtumissa peritään siirron tekevältä käyttäjältä siirtomaksu, ja siirtotapahtuman lohkon ongelman ratkaissut käyttäjä palkitaan siirtomaksulla.

Ethereumin suurin ero Bitcoiniin on, että Ethereum sisältää Turing-täydellisen virtuaalikoneen \cite{buterin2017ethereum}, ja virtuaalikoneen tilaa muuttavat tapahtumat perivät myös maksuna Ether-virtuaalivaluuttaa. Virtuaalikoneelle on mahdollista rakentaa Solidity-ohjelmointikielellä applikaatioita, ja näiden applikaatioiden tilan muuttaminen perii myös maksuna Ether-virtuaalivaluuttaa. 

Koska Ethereum käyttää myös laskennallisesti vaativaa PoW-konsensusmekanismia, sen vuosittainen energiankulutus on 94,73 terawattituntia \cite{ethereumenergy}, joka vastaa Filippiinien vuosittaista energiakulutusta. Tämän lisäksi PoW:n ongelmana on vähäinen määrä transaktioita sekunnissa, joka on tällä hetkellä Ethereumissa vain noin 14 \cite{ethereum-tps}. Yksi siirtotapahtuma kuluttaa Ethereum-lohkoketjussa keskimäärin 207,95 kilowattituntia energiaa. Ethereumissa siirtomaksut (eng. \textit{gas fees}) ovat myös nousseet korkeiksi, mikä on aiheuttanut paljon keskustelua lohkoketjun tasa-arvoisuudesta ja skaalautuvuudesta. Ruuhka-aikoina Ethereum-lohkoketjussa älysopimusinteraktiot ja siirtotapahtumat ovat maksaneet pahimmillaan yli viisikymmentä dollaria \cite{ethereum-fees}.

Lohkoketjujen energiankulutuksesta nousseiden huolien ja pienen transaktiot-per-sekunti (TPS) määrän takia Vitalik Buterin ja Ethereum-tiimi ovat jo 2014 vuodesta alkaen kehittäneet Ethereum 2.0 -projektia, jonka on arvioitu valmistuvan Ethereum Foundationin mukaan aikaisintaan 2022 \cite{eth2.0}. Ethereum 2.0 toimii PoW:n sijaan Proof-of-Stake konsensusmekanismilla. Projektin valmistuminen on  lykkääntynyt kuitenkin jo useasti aikaisemmin. Ethereum 2.0:n TPS:n on arvioitu olevan tuhansia ja energiankulutuksen verrattavissa muihin Proof-of-Stake konsensusmekanismilla toimiviin lohkoketjuihin. Nämä ovat kuitenkin vain projektin tavoitteita, eikä näistä ole olemassa tarkkoja arvioita.

\end{subsection}

\end{otherlanguage}