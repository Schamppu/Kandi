\begin{section}{Elektroniikkajäte\label{elektroniikkajate}}

PoW on elektroniikkajätteen kannalta ongelmallinen. PoW:ssa kaupallisesti kannattavaan louhintaan soveltuu ainoastaan siihen erikoistuneet laitteet (ASIC, Application Specific Integrated Circuit), jotka kaikki joudutaan korvaamaan uudemmilla laitteilla kahden vuoden välein \cite{btc-carbon-ewaste}. Käytöstä poistuvat laitteet tuottavat suuren määrän elektronista jätettä: kesäkuun 2019 ja kesäkuun 2020 välillä Bitcoinin arvioitiin tuottaneen 27,1 kilotonnia elektronista jätettä. Suomi tuotti vuonna 2019 noin 110 kilotonnia elektronista jätettä \cite{ewaste-finland}, johon suhteutettuna Bitcoin vastaisi noin 24,6\% Suomen samaan aikaan tuottamasta elektronisesta jätteestä. Bitcoin-louhijoita miningpoolstats.com-verkkosivun tilastojen mukaan on noin 5,11 miljoonaa \cite{btc-pool-stats-miner-count}.

PoSp:ssä on joulukuussa 2021 miningpoolstats.com-verkkosivun mukaan louhijoita 209 tuhatta \cite{chia-pool-stats}. PoSp:n aiheuttama kulutus laitteiden kovalevyille on kiistelty aihe, eikä siitä ole tehty tarkkoja tutkimuksia. Osa käyttäjistä on kuitenkin raportoinut esimerkiksi SSD:n hajoavan käytössä jo muutaman kuukauden sisällä. Tämän lisäksi Chia aiheutti julkaisunsa aikana uusien kovalevyjen myynnissä merkittävän nousun, ja mikäli Chia kuluttaa saman verran kovalevyjä kuin tyypilliset kaupallisessa käytössä olevat serverit, on niiden elinikä noin neljä vuotta \cite{manufacturingcarbon1}.

Chia kuitenkin eroaa kaupallisista serveritietokoneista siinä, että Chiassa kovalevyille ei tallenneta mitään kriittistä dataa. Tällöin tallennuslaitteita voidaan käyttää niiden hajoamiseen saakka. On siis mahdollista, että Chian louhijat ostaisivat käytettyjä kovalevyjä esimerkiksi kaupallisten servereiden luopuessa niistä takuuajan umpeuduttua. Tällöin Chia edistäisi tallennuslaitteiden kierrättämistä ja toimisi suureksi osaksi tallennuslaitteiden elinkaaren loppusijoituskohteena. Kovalevyjen kysynnän ja hinnan kasvaessa tällainen tilanne on erityisesti mahdollinen, sillä toisin kuin PoW:ssa, Chiassa louhintaa voi tehdä normaalilla kuluttajatason laitteella, eikä ASIC:n käyttö ole tarpeellista.

PoS puolestaan on elektroniikkajätteen määrää vertailtaessa hyvä. PoS-lohkoketjujen elektroniikkajäte tulee solmuja ylläpitävistä tietokoneista. Solmujen määrä tämän tutkielman käsittelemissä lohkoketuissa, eli Cardanossa ja Algorandissa, on huomattavasti pienempi verrattuna PoW- ja PoSp:n louhijoiden määrään. Cardanolla on joulukuussa 2021 noin 4700 osakkuusvarantoa \cite{cardano-staking-pools}, mistä voidaan arvioida solmuja olevan suunnilleen saman verran, kun solmut ylläpitävät osakkuusvarantoja. Algorandin tilastossa joulukuussa 2021 solmuja on noin 1500 \cite{algorand-tps-nodes-etc}. Hyvin optimoiduissa 
PoS-lohkoketjuissa solmun ylläpidon tulisi toimia kuluttajatason tietokoneella, eikä se rasittaisi tietokonetta normaalia käyttöä enempää. Tällainen optimoinnin taso on vielä kuitenkin saavuttamatta, sillä toistaiseksi Cardano ja Algorand vaativat kaupallisen serveritietokoneen tasoisen laitteen solmun ylläpitoon \cite{algorand-energy}. Korkeiden laitteistovaatimusten merkittävin aiheuttaja on näiden lohkoketjujen suuri transaktioiden määrä sekunnissa, mikä tarkoittaa jatkuvaa rasistusta prosessorille sekä tallennustilalle. Solmujen lukumäärän ollessa niin pieni verrattuna PoW- ja PoSp -lohkoketjuihin voidaan kokonaiselektroniikkajätteen kuitenkin olettaa olevan merkittävästi pienempi PoS-lohkoketjuissa, silloinkin vaikka solmujen laitteet kuluisivatkin yhtä nopeasti.

\end{section}