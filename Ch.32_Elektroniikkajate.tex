\section{Elektroniikkajäte\label{elektroniikkajate}}
\begin{otherlanguage}{english}

PoW on elektroniikkajätteen kannalta ongelmallinen. PoW:ssa kaupallisesti kannattavaan louhintaan soveltuu ainoastaan siihen erikoistuneet laitteet (ASIC, Application Specific Integrated Circuit), jotka kaikki joudutaan korvaamaan uudemmilla laitteilla kahden vuoden välein \cite{btc-carbon-ewaste}. Käytöstä poistuvat laitteet tuottavat suuren määrän elektronista jätettä: kesäkuun 2019 ja kesäkuun 2020 välillä Bitcoinin arvioitiin tuottaneen 27,1 kilotonnia elektronista jätettä. Suomi tuotti vuonna 2019 noin 110 kilotonnia elektronista jätettä \cite{ewaste-finland}, johon suhteutettuna Bitcoin vastaisi noin 24,6\% Suomen samaan aikaan tuottamasta elektronisesta jätteestä. Bitcoin-louhijoita mininpoolstats.com-verkkosivun tilastojen mukaan on noin 5,11 miljoonaa \cite{btc-pool-stats-miner-count}.

PoS on puolestaan elektroniikkajätteen määrää vertailtaessa hyvä. PoS-lohkoketjujen elektroniikkajäte tulee solmuja ylläpitävistä tietokoneista. Solmujen määrä on tämän tutkielman käsittelemissä lohkoketuissa, eli Cardanossa ja Algorandissa, silti verrattuna PoW- ja PoSp:n louhijoihin varsin pieni. Cardanolla on joulukuussa 2021 noin 4700 osakkuusallasta \cite{cardano-staking-pools}, mistä voidaan arvioida solmuja olevan suunnilleen saman verran, kun solmut ylläpitävät osakkuusaltaita. Algorandin tilastossa joulukuussa 2021 solmuja on noin 1500 \cite{algorand-tps-nodes-etc}. Hyvin optimoiduissa PoS-lohkoketjuissa solmu vaatii vain tyypillisen kuluttajatason tietokoneen toimiakseen. Solmun ylläpitäminen hyvin optimoidussa PoS-lohkoketjussa ei ole laitteistolle normaalia käyttöä kuluttavampaa, sillä PoS-solmujen ylläpito ei vaadi prosessorin jatkuvaa rasittamista tai jatkuvaa lukemista kovalevyltä. Tämän vuoksi PoS-laitteisto kestää käyttöä pidempään, eikä laitteistoa tule jatkuvasti päivittää paremmaksi säilyttääkseen kilpailukykyisyyden.

PoSp:ssä on miningpoolstats.com-verkkosivun mukaan louhijoita 209 tuhatta joulukuussa 2021 \cite{chia-pool-stats}. PoS:n verrattuna lohkoketjua ylläpitäviä laitteita on jo huomattavasti enemmän. PoSp:n aiheuttama kulutus laitteiden kovalevyille on kiistelty, eikä siitä ole tehty tarkkoja tutkimuksia. Osa käyttäjistä on kuitenkin raportoinut esimerkiksi SSD:n hajoavan käytössä jo muutaman kuukauden sisällä. Tämän lisäksi Chia aiheutti julkaisunsa aikana uusien kovalevyjen myynnin kasvua, ja mikäli Chia kuluttaa saman verran kovalevyjä kuin tyypilliset kaupallisessa käytössä olevat serverit, on niiden elinikä noin neljä vuotta \cite{manufacturingcarbon1}.

Chia kuitenkin eroaa kaupallisista serveritietokoneista siinä, että Chiassa kovalevyille ei tallenneta mitään kriittistä dataa. Tällöin tallennuslaitteita voidaan käyttää niiden hajoamiseen saakka. On siis mahdollista, että Chian louhijat ostaisivat käytettyjä kovalevyjä esimerkiksi kaupallisten servereiden luopuessa niistä takuuajan umpeuduttua. Tällöin Chia edistäisi tallennuslaitteiden kierrättämistä ja toimisi suureksi osaksi tallennuslaitteiden elinkaaren loppusijoituskohteena. Kovalevyjen kysynnän ja hinnan kasvaessa tällainen tilanne on etenkin mahdollinen, sillä toisin kuin PoW:ssa, Chiassa louhintaa voi tehdä normaalilla kuluttajatason laitteella, eikä ASIC:n käyttö ole tarpeellista.

\end{otherlanguage}