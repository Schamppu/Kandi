\chapter{Konsensusmekanismien vertailu\label{vertailu}}

Tämä kappale vertailee konsensusmekanismien kokonaisenergiankulutusta, yksittäisten siirtotapahtumien energiankulutusta sekä muita tärkeitä konsensusmekanismeihin liittyviä ominaispiirteitä, kuten skaalautuvuutta ja turvallisuutta. Vertailu tehdään käyttämällä teoriaosuudessa esiteltyjä merkittävimpiä lohkoketjuja esimerkkitapauksina kunkin konsensusmekanismin tyypillisestä energiankulutuksesta.

\begin{center}
\begin{table}[!hbtp]
\begin{tabular}{   | c |  c |  c |  c | c |   } 
  \hline
 \thead {Lohkoketjun \\ nimi} & \thead {Energiankulutus \\ (vuodessa)} & \thead {Transaktioita \\ sekunnissa (TPS)} & \thead {Energiankulutus \\ (siirtoa kohden)} \\ 
  \hline
 \makecell {Bitcoin (PoW)} & 200,57 TWh \cite{bitcoinenergy} & $\sim$7 \cite{bitcoin-tps} & 2006,54 kWh \\
  \hline
 \makecell {Ethereum (PoW)} & 94,73 TWh \cite{ethereumenergy} & $\sim$14 \cite{ethereum-tps} & 207,95 kWh \\
  \hline
 \makecell {Cardano (PoS)} & 6 GWh \cite{cardanoenergy} & $\sim$257 \cite{cardano-tps} & 0,5479 kWh  \\
  \hline
 \makecell {Algorand (PoS)} & 4,9 GWh \cite{algorandenergy} & $\sim$1300 \cite{algorandenergy} & 0,000008 kWh  \\
  \hline
 \makecell {Chia (PoSp)} & 0,307 TWh \cite{chiaenergy} & $\sim$20\footnotemark & 0,4864 kWh\footnotemark[\value{footnote}]  \\
  \hline
\end{tabular}
\caption{\label{tab:pow-database}Energiankulutus- ja sekunnissa tapahtuvien siirtotapahtumien vertailu eri lohkoketjujen välillä.}
\label{table-energy}
\end{table}
\end{center}

Taulukosta \ref{table-energy} huomataan, että PoW-lohkoketjujen vuosittainen energiankulutus on sekä PoS- että PoSp-konsensusmekanismeilla toimiviin lohkoketjuihin verrattuna merkittävästi suurempi. Vuosittainen energiankulutus on kuitenkin vertailukohteena harhaanjohtava, sillä sitä on hankala käyttää lohkoketjun energiatehokkuuden mittaamiseen. Täten taulukossa \ref{table-energy} on myös esitetty kuinka paljon energiaa yksi siirtotapahtuma vaatii.

\vspace{1cm}

\footnotetext{Siirtotapahtumien määrän on arvioitu olevan samaa luokkaa PoW-konsensusmekanismien kanssa. Ks. kappale \ref{chia}, missä esitetään millä menetelmällä Chian TPS ja energiankulutus siirtotapahtumaa kohden on arvioitu.}

Energiankulutusta vertailtaessa on lohkoketjujen markkina-arvolla myös merkittävä vaikutus niiden energiankulutukseen. Tyypillisesti matalan markkina-arvon lohkoketjut kuluttavat vähemmän energiaa, kun niissä on vähemmän solmuja ja louhijoita verrattuna suurempiin lohkoketjuihin. Kuvasta \ref{fig_energy} näkyy, miten lohkoketjun markkina-arvon noustessa sen energiankulutus kasvaa merkittävästi.

\begin{filecontents}{thu2.dat}
X Arvo  	Kulutus
1 80	    19.87 % 80.19
2 235	    36.75 % 235.13
3 364	    77.78 % 364.23
4 613	    135.12 % 613.34
5 1230  	189.88 % 
\end{filecontents}

\begin{figure}[h!]
\centering
\begin{tikzpicture}
\begin{axis}[
axis lines=middle,
ymin=19.80,
ymax=200,
x label style={at={(current axis.right of origin)},anchor=north, below=8mm},
%title={\textit{\textbf{Bitcoinin energiankulutus markkina-arvon kasvaessa}}},
xlabel=Markkina-arvo (miljardeissa),
ylabel=Energiankulutus vuodessa (TWh),
xticklabel style = {rotate=0,anchor=east,below=1mm},
enlargelimits = false,
ymajorgrids=true,
grid style=dashed,
xticklabels from table={thu2.dat}{Arvo},xtick=data]
\addplot[orange,thick,mark=square*] table [y=Kulutus,x=X]{thu2.dat};
\end{axis}
\end{tikzpicture}
\caption{Bitcoinin energiankulutus markkina-arvon kasvaessa. Kuvaajan tilastot on kerätty ottamalla sen markkina-arvo Coingecko-verkkosivulta ja samalta päivältä sen energiankulutus Diginomist-verkkosivun energiankulutustilastosta Waybackmachinella.}
\label{fig_energy}
\end{figure}

Energiankulutuksen kasvu markkina-arvon kasvaessa kuvan \ref{fig_energy} esittämällä tavalla pätee vain PoW-konsensusmekanismilla toimiviin lohkoketjuihin. PoS:ssä solmujen määrä ei kasva yhtä merkittävästi markkina-arvon kasvaessa, sillä PoS:ssä osakkaiden ja sijoitetun pääoman määrä kasvaa vastaavalla tavalla, mutta solmujen ylläpito ei ole yhtä kannattavaa käyttäjälle. PoSp:ssä puolestaan on esitetty, että lohkoketjun vaatima tallennustilan määrä voi kasvaa vastaavalla tavalla, mutta tallennustilan ollessa energiankulutukseltaan huomattavasti tehokkaampaa ei energiankulutus kasva yhtä merkittävästi.

Tutkielma vertailee seuraavaksi jokaisen tutkielman käsittelemän konsensusmekanismin energiankulutusta ja niiden muita ekologisuuteen vaikuttavia tekijöitä. Näiden asioiden lisäksi tutkielma pohtii konsensusmekanismeihin liittyviä mahdollisia muita tekijöitä, kuten niiden skaalautuvuutta ja hajautuneisuutta.

\section{Proof-of-Work\label{spacemint}}
\begin{otherlanguage}{english}

Vertailua

\end{otherlanguage}
\section{Proof-of-Stake\label{spacemint}}
\begin{otherlanguage}{english}

Vertailua

\end{otherlanguage}
\section{Proof-of-Space\label{spacemint}}
\begin{otherlanguage}{english}

Vertailua

\end{otherlanguage}