\subsection{Algorand\label{algorand0}}
\begin{otherlanguage}{english}

Algorand lohkoketjun kehittäjänä toimii Silvio Micalin vuonna 2017 perustama Algorand, Inc. yritys \cite{algorandwhitepaper}. Algorandin pääverkko julkaistiin 2019, ja lohkoketjuna se pyrkii toimimaan ympäristöystävällisesti hyödyntämällä Proof-of-Stake konsensusmekanismin mahdollistamaa energiatehokkuutta. Tämän lisäksi Algorandin tavoitteena on olla skaalautuva lohkoketju mahdollistaen kymmeniä tuhansia siirtotapahtumia sekunnissa.

Algorand esittää tutkimuksissaan Bysantin kenraaliongelman, ja esittää Proof-of-Stake konsensusmekanismin ratkaisuksi kyseiseen ongelmaan \cite{algorandtech}. Bysantin kenraaliongelma on hyvä esimerkki kaikkien hajautettujen järjestelmien haasteesta, ja onkin hyvin sovellettavissa lohkoketjuihin: osa kenraaleista on pettureita ja osa luotettavia. Kenraaleiden täytyy pyrkiä valmistelemaan hyökkäys, ja mikäli valmistelun tehneet kenraalit ovat pettureita, saattaa päätös olla epäsuotuisa. Näin ollen täytyy pyrkiä saada konsensus luotettavien kenraaleiden kesken, vaikka on mahdotonta tietää ketkä kenraaleista on luotettavia.

Algorand käyttää algoritmista satunnaisuutta Proof-of-Stakessa \cite{algorandtech}. Kuka tahansa käyttäjistä voi sijoittaa valuuttaa (stake) lohkoketjuun, ja on tällöin Bysantin kenraaliongelmassa yksi kenraaleista. Algorand on skaalatuissa testeissä todennut, että mikäli 2/3 lohkoketjuun sijoitetuista varoista on luotettavien kenraalien hallussa, tulee tällöin sen käyttämällä painotetulla satunnaisella valinnalla valituksi luotettavat kenraalit. Painotettu satunnainen valinta Algorandissa toimii niin, että ne käyttäjät joilla on sijoitettuna enemmän varoja tulevat todennäköisemmin valituksi päätöksen tekeviksi kenraaleiksi.

Lohkoketjuissa kuitenkaan ei puhuta kenraaleista, vaan Algorandin tutkimuksissa esittämä kenraaliongelma vastaa Proof-of-Stake konsensusmekanismeissa tyypillistä uusien lohkojen hyväksyjien valintaa joka tehdään sijoitettuja varojen mukaan painotetulla satunnaisella valinnalla, kuten kuvaajassa \ref{fig_pos}. Cardano käyttää myös tällaista valintaperiaatetta \cite{cardanowhitepaper}.

Algorand sisältää myös Cardanon tapaan virtuaalikoneen (Algorand Virtual Machine, AVM), jolla on mahdollista ohjelmoida älysopimuksia ja applikaatioita, jotka hyödyntävät lohkoketjuja. Algorandin energiakulutus yhtä siirtoa kohden on varsin alhainen verratunna muihin lohkoketjuihin johtuen sen teknologian mahdollistamasta transaktiot-per-sekunti (TPS) määrästä: Algorand pystyy jo tällä hetkellä hyväksymään 1300 transaktiota sekunnissa, ja arvioi pystyvänsä tulevaisuudessa skaalaamaan määrän kymmeniin tuhansiin \cite{algorandenergy}. Algorand huhtikuussa 2021 arvioi kuluttaneensa vain 0,000008 kilowattituntia yhtä siirtotapahtumaa kohden, kun taas Bitcoin kulutti samaan aikaan 930 kilowattituntia siirtoa kohden ja Ethereum 70 kilowattituntia. Arvioidulla energiankulutuksellaan siirtotapahtumaa kohden Algorand kuluttaisi keskimäärin vuodessa 4,9 gigawattituntia energiaa. Energiankulutus on kuitenkin mitattu Algorandin kehitystiimin toimesta käyttäen laskennassa ainoastaan optimaalisinta laitteistoa (Raspberry Pi 4), minkä vuoksi todellinen energiankulutus on mahdollisesti huomattavasti suurempi.

\end{otherlanguage}