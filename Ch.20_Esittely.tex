\chapter{Konsensusmekanismien esittely\label{methods}}
\begin{otherlanguage}{english}

Lohkoketjut ovat lista toisiinsa linkitettyjä tietueita, joita kutsutaan lohkoiksi (blocks) \cite{blockchain1}. Lohkot yhdistetään ketjussa olevaan edelliseen lohkoon niiden tiivisteosoittimen (hash pointer), aikaleiman sekä tapahtumadatan (transaction data) perusteella. Lohkoketjujen tärkeimpiin ominaisuuksiin liittyy sen sisältämän datan muuttumattomuus. Lohkoketjut mahdollistaa niiden käyttämät kryptograafiset menetelmät, kuten esimerkiksi SHA-256 tiivistämisalgoritmi.

Tämä kappale esittelee kolme konsensusmekanismia: Proof-of-Work (PoW), Proof-of-Stake (PoS) sekä Proof-of-Space (PoSp). Jokaisesta konsensusmekanismista esitellään niiden toimintaperiaatteet, markkinaosuudeltaan merkittävimmät lohkoketjut ja miten nämä lohkoketjut hyödyntävät kyseistä konsensusmekanismia.


\section{Proof-of-Work (PoW)\label{pow}}
\begin{otherlanguage}{english}
% Oikoluettu: X 
Proof-of-Work, josta tässä tutkielmassa käytetään lyhennettä PoW, on tutkielman kirjoittamisen hetkellä markkinaosuudeltaan suurin konsensusmekanismi. PoW:ta käyttää kaksi suurinta lohkoketjua: Bitcoin ja Ethereum \cite{Coingecko}.

\begin{figure}[!htbp]
\centering
\fbox{\includegraphics[width=0.98\textwidth]{proof-of-work}}
\caption{Esimerkki Proof-of-Work ja Proof-of-Space konsensusmekanismien toimintaperiaatteesta}
\label{fig_pow}
\end{figure}

\vspace{1mm}

PoW vaatii, että sen turvaamiseen osallistuvat käyttäjät, joita kutsutaan louhijoiksi (eng. \textit{miners}), ratkaisevat kryptografisen ongelman varmistaakseen, että lohkoketjuun lisätty lohko on oikea \cite{blockchain1}. Ongelman ratkaisemisen jälkeen oikean ratkaisun löytäneelle käyttäjälle annetaan palkkio (eng. \textit{block reward}). Kaavio \ref{fig_pow} näyttää millaisista vaiheista uusien lohkojen lisääminen koostuu. Tätä kokonaisuutta millä lohkoketjua turvataan kutsutaan louhinnaksi (eng. \textit{mining}) ja tämä tutkielma käyttää myös kyseistä termiä viitatessaan lohkoketjun turvaamisprosessiin. PoW-lohkoketjujen sisältämää dataa ja toimintaa esitellään tutkielman johdannossa taulukossa \ref{table-pow-database}.

Kaavio \ref{fig_pow} esittää, miten uuden lohkon luominen, varmentaminen, ja lisääminen tapahtuu. Tämä kolmeen vaiheeseen (luonti, louhinta, palkitseminen) jaettu prosessi toimii tarkemmin selitettynä seuraavalla tavalla:

\begin{enumerate}
\item Luodaan siirtotapahtumia sisältävä lohko. Taulukossa \ref{table-pow-database} esiteltiin, millaista dataa lohkoketjun lohkot sisältävät. Kaavion \ref{fig_pow} ensimmäisessä vaiheessa luodaan uusi lohko, missä louhinnan kannalta olennaisinta on siitä luotu tiiviste.
\item Louhijat pyrkivät ratkaisemaan lohkon tiivisteeseen liittyvän ongelman. Koska esimerkiksi Bitcoinissa käytetty SHA-256 tiivistealgoritmi luo annetusta merkkijonosta aina saman tiivisteen, tehdään oikean tiivisteen arvaamisesta louhijoille hankalaa lisäämällä tiivisteen alkuun jokin määrä nollia \cite{blockchain1}. Vaikeusaste määräytyy louhijoiden määrän (eli käytännössä kokonaislaskentatehokkuuden) mukaan niin, että uusi lohko varmennetaan noin kymmenessä minuutissa. Lohko varmennetaan niin, että louhijat pyrkivät luomaan tiivisteen, jonka alussa on yhtä monta nollaa kuin lohkon tiivisteessä. Koska yksi merkkijono tuottaa tiivistelagoritmilla aina saman tiivisteen, joutuvat louhijat lisäämään tiivisteeseen satunnaisen luvun, mitä kutsutaan nonceksi (eng. \textit{number once}), jolloin riittävällä määrällä yrityksiä joku louhijoista saa luotua sellaisen tiivisteen, missä on vaikeuasteen vaatima määrä nollia tiivisteen alussa. Tämä on yksinkertaistettuna esitetty kaavion \ref{fig_pow} toisessa vaiheessa.
\item Oikean ratkaisun (eli käytännössä oikean tiivisteen tuottaneen noncen) löytänyt käyttäjä lähettää vastauksensa lohkoketjulle ja vastaanottaa ratkaisun löytämisestä palkkion. Palkkioksi käyttäjä saa jonkin määrän lohkoketjun käyttämää kryptovaluuttaa itselleen. Kaavion \ref{fig_pow} viimeinen vaihe käsittelee palkkionjakoa.
\end{enumerate}

Konsensusmekanismeja vertailtaessa PoW:n eduksi mainitaan usein sen suuri hajautuneisuus. Lohkoketjuissa tyypillisesti periaatteena on, että mitä hajautuneempi lohkoketju on, sitä turvallisempi se on, sillä laajasti hajautuneisiin lohkoketjuihin on käytännössä vaikeaa kohdistaa hyökkäys \cite{51attack}. Esimerkiksi Bitcoinia vastaan hyökkääminen vaatisi, että hyökkääjä saisi käyttöönsä yli puolet Bitcoinin louhintaan käytetystä laskentatehosta. Tästä käytetään Bitcoinin tapauksessa nimitystä "51\% Attack".


Tutkielma esittelee seuraavaksi kaksi suurinta PoW-konsensusmekanismilla toimivaa lohkoketjua: Bitcoinin ja Ethereumin.

\begin{subsection}{Bitcoin\label{bitcoin}}
% Oikoluettu: X 

Bitcoin kehitettiin vuonna 2009 nimimerkillä "Satoshi Nakamoto" esiintyneen tuntemattoman tahon toimesta ja siitä julkaistiin sen toimintaperiaatetta \cite{bitcoin1, satoshibitcoin} kuvaava tutkimusartikkeli, jota kryptovaluuttakeskusteluissa kutsutaan \textit{white paperiksi}. Bitcoinin tavoitteina fiat-rahaan (perinteinen raha, kuten eurot ja dollarit) verrattuna on, että sitä voi siirtää ilman rajoituksia, sen siirtomaksut ovat alhaisia ja että jokaisella käyttäjillä on pääsy täydelliseen tilikirjaan joka sisältää kaikki tapahtumat. Näiden lisäksi valuuttaa turvaa konsensusmekanismi ja kryptografia. Kryptovaluuttana Bitcoin on myös luottamukseton (eng. \textit{trustless}), eli valuutta ei vaadi toimiakseen mitään kolmatta osapuolta johon käyttäjien täytyisi luottaa vaan se toimii hajautetusti. Tavoitteet ovat muilta osin Bitcoinissa toteutuneet, mutta siirtomaksut ovat kryptovaluutan hinnan noustua nousseet jo kymmeniin euroihin.

Bitcoinissa on vain yhdenlaisia lohkoja, eli siirtolohkoja (eng. \textit{transaction block}). Valuuttaa siirtäessään käyttäjä joutuu maksamaan siirtomaksun (eng. \textit{transaction fee}) ja siirtotapahtuma lisätään seuraavaan lohkoon, jota ei ole vielä lisätty osaksi lohkoketjua. Kun lohko lisätään osaksi lohkoketjua se varmennetaan, eli louhija ratkaisee lohkoon liittyvän pulman ja saa palkintona sen sisältämissä siirtotapahtumissa maksetut siirtomaksut sekä lohkopalkkion (eng. \textit{block reward}). Lohkopalkkioiden määrä puolittuu noin joka neljäs vuosi, ja Bitcoineja tulee olemaan kokonaisuudessaan 21 miljoonaa kappaletta \cite{satoshibitcoin}.

Bitcoinissa ongelman ratkaisemiseksi ja palkkion saamiseksi tärkein tekijä on käyttäjän louhintaan käyttämän tietokoneen laskennallinen tehokkuus \cite{bitcoin1}. Bitcoinissa palkkion saannin todennäköisyys on suoraan verrannollinen käyttäjän laskennallisen tehon suhteeseen lohkoketjun kokonaislaskentatehosta. Tästä syystä valtaosa louhijoista liittyykin niin kutsuttuihin louhintavarantoihin (eng. \textit{mining pool}). Louhintavarannoissa louhijat yhdistävät laskentatehonsa ja saavat louhintavarannon voittamista palkkioista osuutensa riippuen siitä, kuinka suuri osuus heidän laskentatehostaan on louhintavarannon kokonaislaskentatehosta.

Louhinnassa palkkion voittamisen todennäköisyyden riippuessa täysin laskennallisesta tehosta Bitcoinin louhinta on aiheuttanut kilpailua laskentatehosta käyttäjien välillä. Laitteistokilpailusta ja johtavasta asemastaan suurimpana kryptovaluuttana Bitcoin vaatiikin paljon energiaa: vuonna 2021 Bitcoinin louhinnan energiankulutuksen on arvioitu olevan 200,57 terawattituntia \cite{bitcoinenergy}, ja yksi siirtotapahtuma kuluttaa keskimäärin noin 2006,54 kilowattituntia energiaa. Mikäli Bitcoin olisi valtio, olisi Bitcoin energiankulutukseltaan suurempi kuin Thaimaa. Bitcoinin energiankulutus onkin herättänyt keskustelua lohkoketjujen ympäristöystävällisyydestä ja ollut suurena vaikuttajana siihen, miksi ympäristöystävällisempiä uusia konsensusmekanismeja on kehitetty.

Bitcoinin skaalautuvuus on myös saanut osakseen arvostelua. Bitcoin pystyy käsittelemään tällä hetkellä ainoastaan seitsemän siirtotapahtumaa sekunnissa \cite{bitcoin-tps}, ja laskettu teoreettinen maksimi Bitcoinia skaalatessa on 27 siirtotapahtumaa sekunnissa mikäli yhden lohkon koko pysyisi yhdessä megatavussa. Verrattuna siihen, että PoS-konsensusmekanismilla toimivat lohkoketjut pystyvät käsittelemään jo nyt kymmenkertaisia määriä siirtotapahtumia sekunnissa \cite{algorandtech, cardano-ouroboros} on Bitcoinin käyttämän PoW-konsensusmekanismin kohtaama kritiikki tehottomana ymmärrettävää.

\end{subsection}
\begin{subsection}{Ethereum\label{ethereum}}

Ethereumin kehitti kanadalais-ukrainalainen Vitalik Buterin, ja se julkaistiin 2014 \cite{buterin2017ethereum}. Ethereum kehitettiin Bitcoinin perustalle, mutta eroaa Bitcoinista radikaalisti siinä, että Ethereum sisältää Turing-täydellisen Ethereum-virtuaalikoneen (Ethereum Virtual Machine, EVM).

Ethereum rakentuu samalle PoW-konsensusmekanille, mitä Bitcoin myös käyttää \cite{buterin2017ethereum}. Louhinta tapahtuu laskennallisesti vaativan ongelman ratkaisulla, mistä oikean ratkaisun löytänyt käyttäjä palkitaan antamalla korvauksena Ether-virtuaalivaluuttaa. Siirtotapahtumissa peritään siirron tekevältä käyttäjältä siirtomaksu, ja siirtotapahtuman lohkon ongelman ratkaissut käyttäjä palkitaan siirtomaksulla.

Ethereumin suurin ero Bitcoiniin on, että Ethereum sisältää Turing-täydellisen virtuaalikoneen \cite{buterin2017ethereum}, ja virtuaalikoneen tilaa muuttavat tapahtumat perivät myös maksuna Ether-virtuaalivaluuttaa. Virtuaalikoneelle on mahdollista rakentaa Solidity-ohjelmointikielellä applikaatioita, ja näiden applikaatioiden tilan muuttaminen perii myös maksuna Ether-virtuaalivaluuttaa. 

Koska Ethereum käyttää myös laskennallisesti vaativaa PoW-konsensusmekanismia, sen vuosittainen energiankulutus on 94,73 terawattituntia \cite{ethereumenergy}, joka vastaa Filippiinien vuosittaista energiakulutusta. Tämän lisäksi PoW:n ongelmana on vähäinen määrä transaktioita sekunnissa, joka on tällä hetkellä Ethereumissa vain noin 14 \cite{ethereum-tps}. Yksi siirtotapahtuma kuluttaa Ethereum-lohkoketjussa keskimäärin 207,95 kilowattituntia energiaa. Ethereumissa siirtomaksut (eng. \textit{gas fees}) ovat myös nousseet korkeiksi, mikä on aiheuttanut paljon keskustelua lohkoketjun tasa-arvoisuudesta ja skaalautuvuudesta. Ruuhka-aikoina Ethereum-lohkoketjussa älysopimusinteraktiot ja siirtotapahtumat ovat maksaneet pahimmillaan yli viisikymmentä dollaria \cite{ethereum-fees}.

Lohkoketjujen energiankulutuksesta nousseiden huolien ja pienen transaktiot-per-sekunti (TPS) määrän takia Vitalik Buterin ja Ethereum-tiimi ovat jo 2014 vuodesta alkaen kehittäneet Ethereum 2.0 -projektia, jonka on arvioitu valmistuvan Ethereum Foundationin mukaan aikaisintaan 2022 \cite{eth2.0}. Ethereum 2.0 toimii PoW:n sijaan Proof-of-Stake konsensusmekanismilla. Projektin valmistuminen on  lykkääntynyt kuitenkin jo useasti aikaisemmin. Ethereum 2.0:n TPS:n on arvioitu olevan tuhansia ja energiankulutuksen verrattavissa muihin Proof-of-Stake konsensusmekanismilla toimiviin lohkoketjuihin. Nämä ovat kuitenkin vain projektin tavoitteita, eikä näistä ole olemassa tarkkoja arvioita.

\end{subsection}

\end{otherlanguage}
\section{Proof-of-Stake (PoS)\label{pos}}

\begin{otherlanguage}{english}\sloppy

Proof-of-Stake, josta tässä tutkielmassa käytetään lyhennettä PoS, sai alkunsa vastauksena PoW:n energiankulutukselle \cite{pos2}. PoS:n ideana on, että lohkoketjun konsensusmekanismi ei perustu laskennalliselle tehokkuudelle, vaan taloudilleselle haitalle mikä aiheutuu louhijoille mikäli lohkoketjua yritetään hyödyntää haitallisessa mielessä. Louhijoita PoS:sa kutsutaan usein osakkaiksi (stakers).

\vspace{1mm}

\begin{figure}[!htbp]
\centering
\fbox{\includegraphics[width=0.98\textwidth]{proof-of-stake}}
\caption{Esimerkki Proof-of-Stake konsensusmekanismin toimintaperiaatteesta}
\label{fig_pos}
\end{figure}

\vspace{1mm}

PoS lohkoketjujen malli koostuu viidestä eri osasta \cite{pos1, pos2}:

\begin{enumerate}
\item Konsensus, missä jokin tietty osuus osakkaista (stakers) ovat yhteisymmärryksessä siitä, että lohkoketjuun tehdyt lisäykset ovat oikeita. Kuvaaja \ref{fig_pos} esittää esimerkin siitä, että konsensus voidaan saavuttaa valitsemalla osakkaista satunnaisella valinnalla neuvosto, joka äänestää uuden lohkon lisäämisestä lohkoketjuun.
\item Osakkuus, eli sijoittamisprosessi, missä käyttäjä voi ryhtyä osakkaaksi sijoittamalla jonkin määrän kryptovaluuttaa lohkoketjuun. Sijoitettu kryptovaluutta lukitaan tyypillisesti joksikin ennalta määrätyksi ajaksi, jolloin sijoitettuja varoja ei voi käyttää.
\item Palkitseminen, missä osakkaille jaetaan sijoitetun valuutan mukaan suhteutettu palkinto siitä, että he ovat olleet osallisina lohkoketjun turvaamisessa hyväksymällä uusia lohkoja. Kuvaaja \ref{fig_pos} esittää esimerkin siitä, miten palkkion saa neuvostoon valitut jäsenet, jotka olivat mukana lohkoketjun varmistamisessa.
\item Rankaiseminen, missä lohkoketjun väärinkäytöstä tai sen yritykseen syyllistyneiden käyttäjien lukitsemista varoista vähennetään jokin osuus. Rankaisuja tapahtuu myös joissakin tapauksissa myös silloin, mikäli osakas valitaan neuvostoon (ks. kuvaaja \ref{fig_pos}) ja jättää äänestämättä tai mikäli neuvostoon valittu osakas yrittää hyväksyä kahta lohkoa yhtäaikaisesti (double signing).
\item Uudelleenvalinta, missä valitaan tyypillisesti painotetun satunnaisuuden perusteella osakkaat, jotka hyväksyvät seuraavan lisäkysen lohkoketjuun. Myös muiden valintaperusteiden käyttö on mahdollista, mutta tyypillisesti valinta tapahtuu painotetulla satunnaisuudella, missä enemmän kryptovaluuttaa sijoittaneet osakkaat todennäköisemmin valikoituvat hyväksymään lisäyksiä lohkoketjuun.
\end{enumerate}

Proof-of-Stake konsensusmekanismin ollaan havaittu sen eri implementaatioissa olevan hyvin vähän energiaa kuluttava. Ongelmia kuitenkin on: matala yksityisyys, pääoman kasautuminen muutamalle osakkaalle ja skaalautuvuus nousevat usein esiin PoS:n kohdistuvassa kritiikissä. Näiden lisäksi lohkoketjujen sisältämää dataa ja infrastruktuuria ylläpitävien solmujen käyttämä laitteisto ja määrä vaikuttavat lohkoketjun sähkönkulutukseen. PoS:n energiankulutuksessa joudutaankin tyypillisesti tasapainottelemaan siinä, että halutaanko korkea hajautuneisuus suurella määrällä solmuja, vai pieni energiankulutus vähemmällä määrällä solmuja jolloin lohkoketjun hajautuneisuus on huonompi.

Seuraavaksi tutkielma esittelee kaksi Proof-of-Stake konsensusmekanismilla toimivaa lohkoketjua: Cardanon ja Algorandin. Cardano-lohkoketju ei ole vielä ottanut älysopimustoiminnallisuuksiaan käyttöön, mutta sen energiatehokkuudesta on jo laajalti tilastoja.

\begin{subsection}{Cardano\label{cardano}}

Cardanon on kehittänyt Charles Hoskinsonin perustama IOHK vuonna 2015 \cite{cardanowhitepaper, iohk}, ja Cardanon sekä sen ekosysteemin kehitystä valvoo itsenäinen Cardano Foundation. 

Cardano käyttää PoS-protokollanaan viiden eri akateemisen instituution yhteistyössä kehittämää Ouroboros-lohkoketjuprotokollaa \cite{cardanowhitepaper}. Ouroboros on PoS-konsensusmekanismia soveltava lohkoketjuprotokolla \cite{cardano-ouroboros}, joka on kehitetty ratkaisuksi Bitcoinin korkeaan energiakulutukseen.

Cardanon tavoitteena on rakentua akateemisesti vertaisarvioitujen tutkimuksien pohjalta, ja sen Ourobos-protokolla mahdollistaa useiden eri ominaisuuksien implementoinnin modulaarisesti \cite{cardanowhitepaper}. Ourobos-protokolla mahdollistaa joustavan kehitysprosessin, jonka ansiosta Cardano lohkoketjuna on skaalattavampi ja tulevaisuudenkestävämpi. Cardanossa on erilaisia satunnaislukugeneraattoreita, mitkä ovat muissa lohkoketjuissa harvinaisia sekä mahdollisuus sivuketjuille (side-chains).

Cardanon käyttämä PoS-konsensusmekanismi on energiakulutukseltaan verrattuna PoW- konsensusmekanismeihin alhainen, sillä PoS ei tarvitse lainkaan laskennalliseen tehokkuuteen tai allokoidun tallennustilan suuruuteen perustuvaa todentamista \cite{cardanowhitepaper}. Cardanon energiakulutuksen on arvioitu 2021 olevan 6 gigawattituntia \cite{cardanoenergy}, missä yhden siirtoon vaadittu energiankulutus on keskimäärin 0.5479 kilowattituntia. Ouroboroksesta 2017 vuonna julkaistu tutkimus osoittaa, että Cardano kykeneen nykymuodossaan käsittelemään noin 257 siirtotapahtumaa sekunnissa \cite{cardano-tps}, mutta tämän oletetaan skaalautuvan satoihin tuhansiin mikäli Cardano implementoi niin kutsutun Hydra-menetelmän \cite{cardano-hydra}. Hydra-menetelmässä lohkoketjun siirtotapahtumia voidaan suorittaa päälohkoketjun lisäksi niin kutsutuissa layer-2 ratkaisuissa, joista kukin kykenee Hydra-menetelmän mukaan suorittamaan noin tuhat siirtotapahtumaa sekunnissa. Layer-2 ratkaisut ovat rinnakkaisesti toimivia lohkoketjuja, jotka hyödyntävät päälohkoketjun, eli layer-1:n, konsensusmekanismia toiminnassaan esimerkiksi älysopimuksella. Tällöin ne jakavat päälohkoketjun turvallisuuden. Nämä eroavat sivuketjuista (eng. \textit{sidechains}) siinä, että sivuketjut käyttävät omaa konsensusmekanismiaan ja toimivat täysin erillisesti päälohkoketjusta.

\end{subsection}
\begin{subsection}{Algorand\label{algorand0}}
\begin{otherlanguage}{finnish}

Algorand-lohkoketjun kehittäjänä toimii Silvio Micalin vuonna 2017 perustama Algorand, Inc. yritys \cite{algorandwhitepaper}. Algorandin pääverkko julkaistiin 2019, ja lohkoketjuna se pyrkii toimimaan ympäristöystävällisesti hyödyntämällä Proof-of-Stake konsensusmekanismin mahdollistamaa energiatehokkuutta. Tämän lisäksi Algorandin tavoitteena on olla skaalautuva lohkoketju mahdollistaen kymmeniä tuhansia siirtotapahtumia sekunnissa.

Lamportin ja kollegoiden vuonna 1982 esittämässä tutkimuksessa \cite{byzantine} esitettiin hajautettuja järjestelmiä käsittelevä niin kutsuttu Bysantin kenraaliongelma. Algorand on pyrkinyt ratkaisemaan kyseisen ongelman, ja esittää Proof-of-Stake konsensusmekanismin ratkaisuksi kyseiseen ongelmaan \cite{algorandtech}. Bysantin kenraaliongelma on hyvä esimerkki kaikkien hajautettujen järjestelmien haasteesta, ja onkin hyvin sovellettavissa lohkoketjuihin: osa kenraaleista on pettureita ja osa luotettavia. Kenraaleiden täytyy pyrkiä valmistelemaan hyökkäys, ja mikäli valmistelun tehneet kenraalit ovat pettureita, saattaa päätös olla epäsuotuisa. Näin ollen täytyy pyrkiä saada konsensus luotettavien kenraaleiden kesken, vaikka on mahdotonta tietää ketkä kenraaleista on luotettavia.

Algorand käyttää algoritmista satunnaisuutta Proof-of-Stakessa \cite{algorandtech}. Kuka tahansa käyttäjistä voi sijoittaa valuuttaa (stake) lohkoketjuun, ja on tällöin Bysantin kenraaliongelmassa yksi kenraaleista. Algorand on skaalatuissa testeissä todennut, että mikäli 2/3 lohkoketjuun sijoitetuista varoista on luotettavien kenraalien hallussa, tulee tällöin sen käyttämällä painotetulla satunnaisella valinnalla valituksi luotettavat kenraalit. Painotettu satunnainen valinta Algorandissa toimii niin, että ne käyttäjät joilla on sijoitettuna enemmän varoja tulevat todennäköisemmin valituksi päätöksen tekeviksi kenraaleiksi.

Lohkoketjuissa kuitenkaan ei puhuta kenraaleista, vaan Algorandin tutkimuksissa esittämä kenraaliongelma vastaa Proof-of-Stake konsensusmekanismeissa tyypillistä uusien lohkojen hyväksyjien valintaa joka tehdään sijoitettuja varojen mukaan painotetulla satunnaisella valinnalla, kuten kuvaajassa \ref{fig_pos}. Cardano käyttää myös tällaista valintaperiaatetta \cite{cardanowhitepaper}.

Algorand sisältää myös Cardanon tapaan virtuaalikoneen (Algorand Virtual Machine, AVM), jolla on mahdollista ohjelmoida älysopimuksia ja applikaatioita, jotka hyödyntävät lohkoketjuja. Algorandin energiakulutus yhtä siirtoa kohden on varsin alhainen verratunna muihin lohkoketjuihin johtuen sen teknologian mahdollistamasta transaktiot-per-sekunti (TPS) määrästä: Algorand pystyy jo tällä hetkellä hyväksymään 1300 transaktiota sekunnissa, ja arvioi pystyvänsä tulevaisuudessa skaalaamaan määrän kymmeniin tuhansiin \cite{algorand-energy-2}. Algorand huhtikuussa 2021 arvioi kuluttaneensa vain 0,000008 kilowattituntia \cite{algorand-energy-2} yhtä siirtotapahtumaa kohden, kun taas Bitcoin kulutti samaan aikaan 930 kilowattituntia siirtoa kohden ja Ethereum 70 kilowattituntia. Arvioidulla energiankulutuksellaan siirtotapahtumaa kohden Algorand kuluttaisi keskimäärin vuodessa 4,9 gigawattituntia energiaa. Energiankulutus on kuitenkin mitattu Algorandin kehitystiimin toimesta käyttäen laskennassa ainoastaan optimaalisinta laitteistoa (Raspberry Pi 4), minkä vuoksi todellinen energiankulutus on mahdollisesti huomattavasti suurempi. Tämän lisäksi Algorandin asettamat minimilaitesuositukset solmun ylläpitämiselle eivät vastaa laskelmissa käytettyä Raspberry Pi 4:sta, sillä minimilaitesuositukset vaativat solmulta viidensadan gigatavun NVMe SSD:n \cite{algorand-min-specs} mitä Raspberry Pi 4:sta ei löydy. Laskelmat on myös tehty oletuksella, että lohkoketju käsittelisi sen maksimäärän verran transaktioita sekunnissa, eli 1300 joka sekunti, vaikka Algorand on vuonna 2021 keskimäärin toiminut alle kymmenesosalla tästä luvusta \cite{algorand-explorer}.

Plattin ja kollegoiden syyskuussa 2021 julkaisemassa tutkimuksessa \cite{algorand-energy} tehtiin arvio Algorandin energiankulutuksesta käyttämällä laskennassa realistisempaa laitteistoa ja sen siirtotapahtumien oikeaa määrää teoreettisen maksimikapasiteetin sijaan. Tällöin laskennassa saavutettiin tulokseksi 0,00534 \cite{algorand-energy} kilowattituntia per transaktio. Plattin ja kollegoiden tutkimus laski, että Algorand suorittaa todellisuudessa keskimäärin 9,85 \cite{algorand-energy} siirtotapahtumaa sekunnissa. Tällöin Algorandin vuosittainen energiankulutus on noin 1,684 gigawattituntia. Tutkielma käyttää tätä lukua vertailussa, sillä aikaisemmin mainituista syistä Algorandin kehittäjätiimin oma arvio on epärealistinen ja kehittäjätiimin oma tutkimus ei ole puolueeton. 

\end{otherlanguage}
\end{subsection}


\end{otherlanguage}
\section{Proof-of-Space (PoSp)\label{posp}}
\begin{otherlanguage}{english}
Proof of Space, josta tässä tutkielmassa käytetään lyhennettä PoSp, kehitettiin ratkaisuna Bitcoinin energiankulutukseen \cite{pospchia1,posp2}. PoSp perustuu siihen, että poiketen PoW:sta louhijat todistavat varanneen tietyn määrän tallennustilaa sen sijaan, että niiden tulisi todistaa käyttäneen tietyn verran laskennallista tehokkuutta ongelman ratkaisulle.

PoSp:n perusajatuksena on, että toisin kuin PoW:ssa, lohkon tiivisteeseen liittyvien ongelmien ratkaisut tallennetaan jo ennen ongelman vastaanottamista louhijan tallennustilaan \cite{posp3spacemint}. Kun ongelma tulee ratkaista, louhijat etsivät tallennustilastaan oikeaa ratkaisua, ja löytäessään ratkaisun validoivat lohkoketjuun tehdyn lisäkysen. Tällöin uuteen lohkoon liittyvän ongelman ratkaisemiseen menee hyvin vähän laskentatehoa, mutta ratkaisujen tallentamiseen vaaditaan paljon tallennustilaa ja aikaa. Näin ollen PoSp:n energiankulutus vastaa käytännössä kuluttajalaitteistolla rakennettua pilvipalvelinta.

Ongelman ratkaisua lukuunottamatta PoSp toimii samalla tavalla kuin PoW: louhija, joka löytää ensimmäisenä oikean ratkaisun lähettää sen muille louhijoille validoitavaksi, ja validoinnin jälkeen oikean ratkaisun löytänyt louhija palkitaan kryptovaluutalla \cite{posp3spacemint}. SpaceMint-projektissa on mainittu PoSp:lle kolme etua verrattuna PoW:n:

\begin{enumerate}
\item Ekologisuus. PoSp tarvitsee toimintaansa tallennustilaa ja suorittaa hakuja tallennustilasta, joka on laskennallisesti huomattavasti edullisempaa kuin PoW:ssa prosessorin laskentatehoa hyödyntävä tiivistealgoritmin tuottamien ongelmien ratkaiseminen
\item Ekonomisuus. Monilla kuluttajatason päätelaitteilla on huomattava määrä käyttämätöntä levytilaa, joka on taloudellisesti katsoen täysin hyödyntämätöntä. PoW:ssa on taloudellista järkevää suorittaa louhintaa vain tietyillä laitteilla ja vain tietyissä tilanteissa, kun kryptovaluutan hinta kattaa sähkön kulutuksen vaatimat kulut.
\item Tasa-arvoisuus. PoW:ssa louhintaa tehdään lähes ainoastaan siihen erikoistuneissa louhintafarmeissa, kun taas yksittäisen henkilön mahdollisuudet tehdä taloudellisesti järkevää louhintaa on pienet. PoSp:ssa on teoreettisesti mahdollista, että louhintaa on mahdollista tehdä kuluttajatason laitteistolla.
\end{enumerate}

PoSp:ssa on matalasta energiakulutuksestaan huolimatta useita ongelmia, minkä takia konsensusmekanismi on marginaalinen verrattuna PoW:n ja PoS:n. Suurin PoSp lohkoketju, Chia, on tällä hetkellään kryptovaluuttojen markkina-arvotilastossa 244. suurin \cite{Coingecko}. Tämä tutkielma esittelee kaksi PoSp konsensusmekanismia hyödyntävää lohkoketjua: Spacemintin ja Chian. Spacemint-lohkoketjua ei ole otettu koskaan käyttöön, mutta se loi perustavanlaatuisen tutkimuksen PoSp:n käytölle.


\subsection{SpaceMint\label{spacemint}}
\begin{otherlanguage}{english}

SpaceMint

\end{otherlanguage}
\subsection{Chia\label{chia}}
\begin{otherlanguage}{english}\sloppy

Chia-lohkoketjua kehittää Chia Network Inc. yritys, jonka perusti Bram Cohen 2017 \cite{chia1}. Bram Cohen on tullut tunnetuksi
BitTorrent-sovelluksesta. Chia-lohkoketjun pääverkko (mainnet) julkaistiin 2021 neljän vuoden testausjakson jälkeen.

Chia käyttää konsensusmekanisminaan Proof of Space and Timea (PoST) \cite{chia1}. Tämä konsensusmekanismi rakennettiin hyvin samanlaiseksi kuin miten Bitcoinin PoW-konsensus-mekanismi toimii, mutta PoW:n energiakulutusongelma ratkaistaan Chiassa käyttämällä Proof-of-Spacea PoW:n sijaan. Chiassa vaaditaan myös vähintään 51\% konsensus, jotta uusi lohko voidaan hyväksyä osaksi lohkoketjua. Chian etuina Bitcoiniin verrattuna on myös Ethereumin kanssa hyvin samanlainen CLVM (Chia Lisp Virtual Machine). CLVM:llä on mahdollista kirjoittaa samaan tapaan kuin Ethereumissa älysopimuksia (smart contracts), millä voidaan tehdä lohkoketjussa toimivia sovelluksia.

Toimintaperiaatteena Chiassa on tämän kappaleen alussa esitellyn kaltaista konsensusmekanismia hyvin vastaava mekanismi \cite{pospchia1}. Käytännössä siis siinä määrin missä Bitcoin vaatii louhijaa todistamaan käyttämänsä laskentatehon, Chiassa vaaditaan louhijaa todistamaan lohkoketjulle allokoitu tallennustila. Chiassa tämä on tehty niin, että louhija tallentaa ratkaisuja tallennustilaansa, ja kun uusi lohko lisätään lohkoketjuun, tulee louhijan löytää tallennetuista ratkaisuista oikea, ja nopeiten oikean ratkaisun löytänyt käyttäjä palkitaan kryptovaluutalla.

Chian energiankulutus on PoW:lla vastaavanlaisiin (Bitcoin, Ethereum) verrattuna verrattain alhainen \cite{chiaenergy}. Koska Chia ei vaadi konsensusmekanismissaan laskentatehoa ja myös kuluttajatason tallennuslaitteiden energiakulutus on alhainen, on Chian arvioitu kuluttavan vuonna 2021 vain 0,307 terawattituntia energiaa. Chian teoreettisesta siirtotapahtumien määrän maksimista sekunnissa ei ole tehty tarkkaa tutkimusta, mutta koska Chian käyttämä PoSp on hyvin samankaltainen PoW:n kanssa voidaan olettaa siirtotapahtumien määrän sekunnissa olevan maksimissaan noin kaksikymmentä per sekunti. Chian kehittäjä Bram Cohen on myös todennut vuoden 2021 maaliskuussa tehdyssä haastattelussa \cite{bram-cohen-interview} siirtotapahtumien maksimimäärän sekunnissa olevan noin kaksikymmentä. Tämä esitetty arvio voidaan vahvistaa sillä, että Chia kohtasi niin kutsutun "dust storm" -hyökkäyksen lokakuussa 2021, jolloin lohkoketju joutui toimimaan maksimikapasiteetillaan ja tällöin saavutettiin hetkellisesti 25 siirtotapahtumaa sekunnissa, mikä on hieman enemmän kuin arvioitu kapasiteetti \cite{chia-tps}. Chian energiankulutus siirtotapahtumaa kohden on arvioitu käyttämällä tätä lukua ja vuosittaista energiankulutusta tarkempien tilastojen puuttuessa. Näiden lukujen perusteella voidaan laskea Chian energiankulutuksen yhtä siirtotapahtumaa kohden olevan 0,4864 kilowattituntia seuraavalla kaavalla:
\begin{mycapequ}[!htbp]
\begin{equation}
\begin{split}
E_{kok} & = 0,307~TWh \\
 & = 307000000~kWh \\
TX_{kok} & = 20~tx/s \\ 
 & = 631138520~tx/a \\
\frac{E_{kok}}{TX_{kok}} & = \frac{307000000~kWh}{631138520~tx/a} \\
 & = 0,4864~kWh/tx
\end{split}
\end{equation}
\caption{2.1: Lasketaan Chian energiankulutus siirtotapahtumaa kohden. \(E_{kok}\) vastaa Chian kokonaisenergiankulutusta vuodessa, kun taas \(TX_{kok}\) vastaa siirtotapahtumien määrää vuodessa, mikäli Chia toimisi sen maksimikapasiteetilla \(20~tx/s\).}
\end{mycapequ}

Energiankulutus ei kuitenkaan ole lohkoketjujen ekologisuutta vertailtaessa ainut olennainen mittari Chian tapauksessa. Chia-lohkoketjun louhijat ovat joulukuussa 2021 allokoineet 34,7 eksbitavua tallennustilaa yhteensä louhinnalle, mikä vastaa noin neljää miljoonaa kymmenen teratavun kovalevyä \cite{chiaspaceusage}. Euroopan unionin 2015 vuoden tutkimuksen mukaan yhden kovalevyn valmistamiseen kuluu kaksi kilowattia energiaa \cite[p~49--50]{manufacturingcarbon1} ja yhden kaupallisessa käytössä toimivan serveritietokoneen käyttöiän on tutkimuksessa laskettu olevan neljä vuotta. Tämän lisäksi kovalevyjen komponenteista kierrätetään 47\%, ja näiden kierrätettyjen komponenttien valmistamiseen käytetyistä materiaaleista ainoastaan 6,3\% voidaan kierrättää. Kierrätetyistä materiaaleista valtaosa on alumiinia, kun taas kriittisisiä raaka-aineita mitä käytetään kovalevyn piirilevyn ja magneettien valmistukseen ei kierrätetä.

Chia on kuitenkin esittänyt, että lohkoketjun louhijat tulisivat käyttämään tallennustilanaan kaupallisessa käytössä toimineiden servereiden päivityksessä poistuneita tallennuslaitteita \cite{chia-mining}. Väitteessä perusteluksi on annettu, että käytettyjen tallennuslaitteiden ostaminen on louhijoille huomattavasti halvempaa. Koska PoSp:ssa ei tallenneta tallennustilaan mitään kriittistä tietoa voidaan tallennuslaitteita käyttää niiden hajoamiseen saakka.

\end{otherlanguage}

\end{otherlanguage}
\end{otherlanguage}