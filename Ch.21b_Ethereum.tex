\begin{subsection}{Ethereum\label{ethereum}}

Ethereumin kehitti kanadalais-ukrainalainen Vitalik Buterin, ja se julkaistiin 2014 \cite{buterin2017ethereum}. Ethereum kehitettiin Bitcoinin perustalle, mutta eroaa Bitcoinista radikaalisti siinä, että Ethereum sisältää Turing-täydellisen Ethereum-virtuaalikoneen (Ethereum Virtual Machine, EVM).

Ethereum rakentuu samalle PoW-konsensusmekanille, mitä Bitcoin myös käyttää \cite{buterin2017ethereum}. Louhinta tapahtuu laskennallisesti vaativan ongelman ratkaisulla, mistä oikean ratkaisun löytänyt käyttäjä palkitaan antamalla korvauksena Ether-virtuaalivaluuttaa. Siirtotapahtumissa peritään siirron tekevältä käyttäjältä siirtomaksu, ja siirtotapahtuman lohkon ongelman ratkaissut käyttäjä palkitaan siirtomaksulla.

Ethereumin suurin ero Bitcoiniin on, että Ethereum sisältää Turing-täydellisen virtuaalikoneen \cite{buterin2017ethereum}, ja virtuaalikoneen tilaa muuttavat tapahtumat perivät myös maksuna Ether-virtuaalivaluuttaa. Virtuaalikoneelle on mahdollista rakentaa Solidity-ohjelmointikielellä applikaatioita, ja näiden applikaatioiden tilan muuttaminen perii myös maksuna Ether-virtuaalivaluuttaa. 

Koska Ethereum käyttää myös laskennallisesti vaativaa PoW-konsensusmekanismia, sen vuosittainen energiankulutus on 94,73 terawattituntia \cite{ethereumenergy}, joka vastaa Filippiinien vuosittaista energiakulutusta. Tämän lisäksi PoW:n ongelmana on vähäinen määrä transaktioita sekunnissa, joka on tällä hetkellä Ethereumissa vain noin 14 \cite{ethereum-tps}. Yksi siirtotapahtuma kuluttaa Ethereum-lohkoketjussa keskimäärin 207,95 kilowattituntia energiaa. Ethereumissa siirtomaksut (eng. \textit{gas fees}) ovat myös nousseet korkeiksi, mikä on aiheuttanut paljon keskustelua lohkoketjun tasa-arvoisuudesta ja skaalautuvuudesta. Ruuhka-aikoina Ethereum-lohkoketjussa älysopimusinteraktiot ja siirtotapahtumat ovat maksaneet pahimmillaan yli viisikymmentä dollaria \cite{ethereum-fees}.

Lohkoketjujen energiankulutuksesta nousseiden huolien ja pienen transaktiot-per-sekunti (TPS) määrän takia Vitalik Buterin ja Ethereum-tiimi ovat jo 2014 vuodesta alkaen kehittäneet Ethereum 2.0 -projektia, jonka on arvioitu valmistuvan Ethereum Foundationin mukaan aikaisintaan 2022 \cite{eth2.0}. Ethereum 2.0 toimii PoW:n sijaan Proof-of-Stake konsensusmekanismilla. Projektin valmistuminen on  lykkääntynyt kuitenkin jo useasti aikaisemmin. Ethereum 2.0:n TPS:n on arvioitu olevan tuhansia ja energiankulutuksen verrattavissa muihin Proof-of-Stake konsensusmekanismilla toimiviin lohkoketjuihin. Nämä ovat kuitenkin vain projektin tavoitteita, eikä näistä ole olemassa tarkkoja arvioita.

\end{subsection}