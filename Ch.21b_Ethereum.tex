\subsection{Ethereum\label{ethereum}}
\begin{otherlanguage}{english}

Ethereumin kehitti kanadalais-ukrainalainen Vitalik Buterin, ja se julkaistiin 2014 \cite{buterin2017ethereum}. Ethereum kehitettiin Bitcoinin perustalle, mutta eroaa Bitcoinista radikaalisti siinä, että Ethereum sisältää Turing-täydellisen Ethereum-virtuaalikoneen (Ethereum Virtual Machine, EVM).

Ethereum rakentuu samalle PoW-konsensusmekanille, mitä Bitcoin myös käyttää \cite{buterin2017ethereum}. Louhinta tapahtuu laskennallisesti vaativan ongelman ratkaisulla, mistä oikean ratkaisun löytänyt käyttäjä palkitaan antamalla korvauksena Ether-virtuaalivaluuttaa. Siirtotapahtumissa peritään siirron tekevältä käyttäjältä siirtomaksu, ja siirtotapahtuman lohkon ongelman ratkaissut käyttäjä palkitaan siirtomaksulla.

Ethereumin suurin ero Bitcoiniin on, että Ethereum sisältää Turing-täydellisen virtuaalikoneen \cite{buterin2017ethereum}, ja virtuaalikoneen tilaa muuttavat tapahtumat perivät myös maksuna Ether-virtuaalivaluuttaa. Virtuaalikoneelle on mahdollista rakentaa Solidity-ohjelmointikielellä applikaatioita, ja näiden applikaatioiden tilan muuttaminen perii myös maksuna Ether-virtuaalivaluuttaa. 

Koska Ethereum käyttää myös laskennallisesti vaativaa PoW-konsensusmekanismia, sen vuosittainen energiankulutus on 81.48 terawattituntia \cite{ethereumenergy}.

\end{otherlanguage}