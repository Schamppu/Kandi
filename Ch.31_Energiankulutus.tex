\begin{section}{Energiankulutus\label{energiankulutus}}

Proof-of-Work (PoW) on taulukon \ref{table-energy} perusteella energiankulutukseltaan ongelmallisin. Ekologiselta kannalta tarkasteltuna PoW-lohkoketjut jo pelkän vaatimansa energian takia eivät ole kestäviä. Kuvaajasta \ref{fig_energy} huomataan, että niiden energiankulutus tulee vain jatkamaan kasvuaan mitä suuremmaksi lohkoketjujen markkina-arvo kasvaa. Tämä energiakulutuksen kasvaminen lohkoketjun markkina-arvon nousun myötä selittyy sillä, että kalliimpaa kryptovaluuttaa voidaan louhia suuremmalla energiankulutuksella ilman, että louhinnasta tulee epäkannattavaa. Mikäli kryptovaluutan hinta laskee joutuu puolestaan osa louhijoista lopettamaan louhimisen, sillä louhinnasta saadut palkkiot eivät enää riitä maksamaan sähkökuluja.

Proof-of-Space (PoSp) pärjää vertailussa huomattavasti paremmin, ja yhtä siirtotapahtumaa kohden vie vähemmän energiaa kuin Cardano. Kuitenkin Chian markkina-arvon ollessa 0,021\% Bitcoinin markkina-arvosta ja 0,48\% Cardanon markkina-arvosta joulukuussa 2021 \cite{Coingecko}, voidaan olettaa lohkoketjun energiankulutuksen sen markkina-arvon kasvaessa nousevan vielä huomattavasti suuremmaksi. Tilastossa on kuitenkin huomioitava se, että Chian energiankulutus siirtotapahtumaa kohden on arvioitu sen mukaan, mikäli lohkoketju toimisi sen maksimikapasiteetilla siirtotapahtumien käsittelyssä sekunnissa. Todellisuudessa Chia käsittelee huomattavasti vähemmän siirtotapahtumia sekunnissa, ja näin ollen energiankulutus siirtotapahtumaa kohden on tilastoissa ilmoitettua lukua suurempi.

Proof-of-Stake (PoS) on vertailussa energiankulutukseltaan tehokkain. Ottaen huomioon, että Cardano on markkina-arvoltaan kuudenneksi suurin, ja kuluttaa silti vain 0,5479 kilowattia yhtä siirtotapahtumaa kohden, suoriutuu se tehokkuusvertailussa hyvin. Algorandin energiankulutus siirtotapahtumaa kohden on vain 0,00534 kilowattituntia, mikä on huomattavasti vähemmän kuin minkään muun vertailtavan lohkoketjun. Tämä korkea energiatehokkuus selittyy sillä, että Algorand käsittelee paljon enemmän siirtotapahtumia sekunnissa kuin muut vertailtavat lohkoketjut \cite{algorand-energy-2}. Algorand käyttää myös tapahtuman varmentamiseen vain hieman yli sataa \cite{algorand-energy} varmentajasolmua, mikä näkyy energiatehokkuutena, mutta vastaavasti voi vaikuttaa negatiivisesti lohkoketjun turvallisuuteen ja hajautuneisuuteen.

\end{section}