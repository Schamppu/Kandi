\section{Energiankulutus\label{energiankulutus}}
\begin{otherlanguage}{english}

Proof-of-Work (PoW) on taulukon \ref{table-energy} perusteella energiankulutukseltaan ongelmallisin. Myös kuvan \ref{fig_energy} mukaan niiden energiankulutus tulee vain jatkamaan kasvuaan, mitä suuremmaksi lohkoketjujen markkina-arvo kasvaa. Ekologisesti jo pelkän energiankulutuksen kautta tarkasteltuna PoW-lohkoketjut eivät ole kestäviä.

Proof-of-Space (PoSp) pärjää vertailussa huomattavasti paremmin, ja yhtä siirtotapahtumaa kohden vie vähemmän energiaa kuin Cardano. Kuitenkin Chian markkina-arvon ollessa 0,021\% Bitcoinin markkina-arvosta ja 0,48\% Cardanon markkina-arvosta joulukuussa 2021 \cite{Coingecko}, voidaan olettaa lohkoketjun energiankulutuksen sen markkina-arvon kasvaessa nousevan vielä huomattavasti suuremmaksi. Tilastossa on kuitenkin huomioitava se, että Chian energiankulutus siirtotapahtumaa kohden on arvioitu sen mukaan, mikäli lohkoketju toimisi sen maksimikapasiteetilla siirtotapahtumien käsittelyssä sekunnissa. Todellisuudessa Chia käsittelee huomattavasti vähemmän siirtotapahtumia sekunnissa, ja näin ollen energiankulutus siirtotapahtumaa kohden on huomattavasti suurempi.

Proof-of-Stake (PoS) on vertailussa energiankulutukseltaan tehokkain. Ottaen huomioon, että Cardano on markkina-arvoltaan kuudenneksi suurin, ja kuluttaa silti vain 0,5479 kilowattia yhtä siirtotapahtumaa kohden, suoriutuu se tehokkuusvertailussa hyvin. Algorandin energiankulutus siirtotapahtumaa kohden on huomattavasti pienempi kuin minkään muun vertailtavan lohkoketjun, mikä osittain selittyy sillä että se käsittelee paljon enemmän siirtotapahtumia sekunnissa kuin mikään muu vertailtava lohkoketju. Energiankulutus Algorandin osalta on kuitenkin laskettu Algorandin kehitystiimin toimesta käyttäen laskennassa ainoastaan optimaalisinta laitteistoa (Raspberry Pi 4), minkä vuoksi energiankulutus todellisuudessa siirtotapahtumaa kohden on merkittävästi suurempi.

\end{otherlanguage}