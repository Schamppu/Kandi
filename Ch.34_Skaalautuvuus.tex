\begin{section}{Skaalautuvuus\label{skaalautuvuus}}

PoW-lohkoketjut ja PoSp-lohkoketjuna Chia häviävät PoS-lohkoketuille siirtotapahtumien määrässä sekunnissa. Taulukon \ref{table-energy} mukaan Bitcoin kykenee ainoastaan seitsemään siirtotapahtumaan sekunnissa, kun taas Ethereum pystyy suorittamaan 14 siirtotapahtumaa sekunnissa. Chian on arvioitu pystyvän suorittamaan noin 20 siirtotapahtumaa sekunnissa. Kun lohkoketjut pystyvät suorittamaan vain hyvin rajallisen määrän siirtoja sekunnissa, on tyypillistä että siirtotapahtumien käsittelyssä menee kauan ja lohkoketju voi ruuhkautua niin, että siirtotapahtumat eivät suoritu lainkaan. Mikäli lohkoketjujen pyrkimyksenä on toimia globaalina virtuaalivaluuttana vaaditaan tehokkaampia ratkaisuja. PoW-lohkoketjuja on kuitenkin mahdollista skaalata nostamalla niiden lohkojen kokoa (eng. \textit{block size}), mikä puolestaan vaatii tehokkaampia solmuja. Tästä syystä tässä tutkielmassa vertailtavat PoW- ja PoSp-lohkoketjut ovat päätyneet lohkoketjun tasa-arvoisuuden ja hajautuneisuuden kannalta siihen tulokseen, että ihanteellinen siirtotapahtumien määrä sekunnissa on noin kaksikymmentä. Mikäli PoW- ja PoSp-lohkoketjut nostaisivat lohkojensa kokoa, joutuisvat useat solmut lopettaa toimintansa laitteistovaatimuksien kasvaessa. Korkeat laitteistovaatimukset saattaisivat myös näkyä suurempana energiankulutuksena. PoW- ja PoSp-lohkoketjut käyttävätkin skaalaamiseensa tyypillisesti layer-2 ratkaisuja.

PoS-lohkoketjuista Cardano ylittää PoW- ja PoSp-lohkoketjujen siirtotapahtumien määrän sekunnissa yli kymmenkertaisesti. Algorand puolestaan ylittää Bitcoinin siirtotapahtumien määrän sekunnissa yli satakertaisesti, ja pärjää vertailussa parhaiten. Näiden lukujen perusteella PoS on konsensusmekanismeista skaalautuvin.

PoW- ja PoSp-lohkoketjuihin on kuitenkin esitetty niiden skaalautuvuuden ratkaisemiseksi jo aikaisemmin mainittuja layer-2 ratkaisuja. Näissä ratkaisussa lohkoketjulle rakennetaan rinnakkaisia lohkoketjuja, missä voidaan suorittaa siirtoja rinnakkaisesti päälohkoketjun, eli niin kutsun layer-1:n kanssa. Layer-2 ratkaisut jakavat päälohkoketjunsa konsensusmekanismin tavallisesti hyödyntämällä älysopimuksia, ja tällä tavoin pystyvät hyödyntämään päälohkoketjun tarjoamaa turvallisuutta. Esimerkiksi Ethereumia on tällä tavoin jo skaalattu Arbitrum- ja Optimism layer-2 ratkaisuilla. Samanlainen ratkaisu pystytään kuitenkin tekemään myös PoS-lohkoketjuissa, ja Cardano on arvioinut pystyvänsä layer-2 ratkaisulla skaalamaan siirtotapahtumien määrää sekunnissa yli miljoonaan \cite{cardano-hydra}. Layer-2 ratkaisujen lisäksi esimerkiksi Ethereumia on jo skaalattu myös sivuketjuilla (eng. \textit{sidechains}), kuten Polygonilla. Sivuketjut eroavat layer-2 ratkaisuista siinä, että ne ovat päälohkoketjusta täysin irrallisia lohkoketjuja, joilla on oma konsensusmekanismi ja tyypillisesti myös oma kryptovaluuttansa. Layer-2 ratkaisuilla ja sivulohkoketjuilla voi mahdollisesti olla merkittäviä energiankulutus- ja ympäristövaikutuksia, mutta näistä on toistaiseksi hyvin vähän tutkimustietoa.

Keskitetyt kryptovaluutanvaihtopalvelut, kuten Binance ja Coinbase, helpottavat myös lohkoketjujen ruuhkautumista. Mikäli kryptovaluuttaa siirretään keskitetyissä vaihdantapalveluissa, ei tällaiset siirrot tapahdu lohkoketjussa itsessään. Kryptovaluuttaa voi myös siirtää muissa lohkoketjuissa, ja tällaisia ratkaisuja on esimerkiksi Wrapped Bitcoin. Wrapped Bitcoinissa myönnetään Bitcoinia vastaava niin kutsuttu wrapped-kolikko, jota voidaan natiivisti siirtää esimerkiksi Ethereumissa tai muissa lohkoketjuissa käyttäjien välillä. Wrapped-kolikot voi halutessaan muuntaa takaisin Bitcoiniksi. Sekä keskitetyt vaihdantapalvelut että wrapped-kolikot vaativat kuitenkin jonkin erillisen tahon, johon käyttäjien täytyy luottaa. Tällöin kryptovaluuttojen ajatus luottamuksettomasta valuutasta kärsii.

Skaalautuvuuden vertailu lohkoketjujen välillä on ongelmallista, sillä kaikkia vertailussa olevia lohkoketjuja on mahdollista skaalata. PoW- ja PoSp-lohkoketjut pärjäävät huonosti vertailussa, mutta niitä voitaisiin skaalata hyvinkin yksinkertaisesti lohkojen kokoa nostamalla. Kuitenkin näiden lohkoketjujen suurimmaksi eduksi usein mainitaan niiden korkea hajautuneisuus, mikä kärsisi tällaisesta muutoksesta merkittävästi. Lohkoketjuja onkin täten mielekkäämpää skaalata esimerkiksi sivulohkoketjuilla ja layer-2 ratkaisuilla, mitkä eivät vaikuta lohkoketjujen hajautuneisuuteen negatiivisesti, ja PoW-lohkoketjuja onkin jo skaalattu tällaisilla ratkaisuilla. Skaalamisen vaikutuksista energiankulutukseen ei ole vielä olemassa merkittäviä tutkimuksia ja sen aiheuttamat muut ympäristöongelmat ovat toistaiseksi epäselvät, mikä vaikeuttaa skaalautuvuuden vertailua tutkielman ekologisuuskeskeisessä kontekstissa.

\end{section}