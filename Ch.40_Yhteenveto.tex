\begin{chapter}{Yhteenveto\label{yhteenveto}}
\begin{otherlanguage}{finnish}

Tutkielmassa esiteltiin kolmea eri konsensusmekanismia: Proof-of-Work (PoW), Proof-of-Stake (PoS) ja Proof-of-Space (PoSp). Tutkielma vertaili näiden konsensusmekanismien välisiä eroja. Konsensusmekanismien ympäristövaikutuksia verratessa havaittiin, että PoW-lohkoketjujen ympäristövaikutukset ovat tällä hetkellä huolestuttavia. Suurimmat lohkoketjut, eli Bitcoin ja Ethereum, eivät ole ekologisesti kestäviä toimiessaan erikoistunutta laitteistoa vaativalla ja energiankulutukseltaan korkealla PoW-konsensusmekanismilla. Uudet konsensusmekanismit, kuten tutkielmassa esitellyt PoS ja PoSp kuitenkin näyttävät lupaavia tuloksia siitä, että hajautetut lohkoketjut voivat olla ympäristön kannalta kestävämpiä. Näistä etenkin PoS osoittautui vertailussa ekologiseksi, vaatien muihin konsensusmekanismeihin verrattuna marginaalisen määrän energiaa (ks. taulukko \ref{table-energy}). PoS:n ei myöskään havaittu kasvattavan kysyntää laitteistolle.

\section{Jatkotutkimusehdotukset\label{jatkotutkimus}}
\begin{otherlanguage}{english}

\end{otherlanguage}
\end{otherlanguage}
\end{chapter}