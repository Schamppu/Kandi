\chapter{Yhteenveto\label{yhteenveto}}

Tutkielmassa esiteltiin kolmea eri konsensusmekanismia: Proof-of-Work (PoW), Proof-of-Stake (PoS) ja Proof-of-Space (PoS). Tutkielma vertaili näiden konsensusmekanismien välisiä eroja  Konsensusmekanismien ympäristövaikutuksia verratessa havaittiin, että PoW-lohkoketjujen ympäristövaikutukset ovat tällä hetkellä huolestuttavia. Suurimmat lohkoketjut, eli Bitcoin ja Ethereum, eivät ole ekologisesti kestäviä toimiessaan erikoistunutta laitteistoa vaativalla ja energiankulutukseltaan korkealla PoW-konsensusmekanismilla. Uudet konsensusmekanismit, kuten tutkielmassa esitellyt PoS ja PoSp kuitenkin näyttävät lupaavia tuloksia siitä, että hajautetut lohkoketjut voivat olla ympäristön kannalta kestävämpiä. Näistä etenkin PoS osoittautui vertailussa ekologiseksi, vaatien muihin konsensusmekanismeihin verrattuna marginaalisen määrän energiaa (ks. taulukko \ref{table-energy}). PoS:n ei myöskään havaittu kasvattavan kysyntää laitteistolle.

\section{Jatkotutkimusehdotukset\label{jatkotutkimus}}
\begin{otherlanguage}{english}

\end{otherlanguage}

Tutkielmassa havaittiin, että vaikka PoW:n ympäristövaikutuksista on laajalti tutkimuksia, on PoSp:n ja PoS:n ympäristövaikutuksista hyvin rajallisesti tutkimuksia. PoSp:ssä kovalevyjen valmistus- ja materiaalikustannukset voivat koitua ongelmallisiksi, mikäli konsensusmekanismi nostaa kysyntää uusille kovalevyille. PoSp:n vaativuudesta kovalevyille olisi hyödyllistä saada lisätietoa louhintatoiminnan kontekstissa.

Uusista konsensusmekanismeista, eli PoS:tä ja PoSp:stä laajempien ympäristövaikutusta käsittelevien tutkimusten puuttuessa vankempien tulosten saamiseksi jatkotutkimusten tekeminen aiheesta on suositeltavaa. PoS:n ja PoSp:n ympäristövaikutuksista ei myöskään niiden skaalautuessa ole varmuutta, joten tämän tutkielman ilmoittamat ympäristövaikutukset voivat olla irrelevantteja mikäli lohkoketjujen käyttöaste kasvaa tulevaisuudessa. Tutkielmassa havaittiin, että ympäristövaikutusten kasvamisen tutkiminen lohkoketjujen skaalautuessa vaatii lisää tutkintaa, jotta voidaan saada paremmin vertailtavia tuloksia niiden ekologisuudesta.