\section{Proof-of-Stake (PoS)\label{pos}}

\begin{otherlanguage}{english}

Proof-of-Stake, josta tässä tutkielmassa käytetään lyhennettä PoS, sai alkunsa vastauksena PoW:n energiankulutukselle \cite{pos2}. PoS:n ideana on, että lohkoketjun konsensusmekanismi ei perustu laskennalliselle tehokkuudelle, vaan taloudilleselle haitalle mikä aiheutuu louhijoille mikäli lohkoketjua yritetään hyödyntää haitallisessa mielessä. Louhijoita PoS:sa kutsutaan usein osakkaiksi (stakers).

PoS lohkoketjujen malli koostuu viidestä eri osasta \cite{pos1, pos2}:

\begin{enumerate}
\item Konsensus, missä kaikki osakkaat (stakers) ovat yhteisymmärryksessä siitä, että lohkoketjuun tehdyt lisäykset ovat valideja.
\item Osakkuus, eli sijoittamisprosessi, missä käyttäjä voi ryhtyä osakkaaksi sijoittamalla jonkin määrän kryptovaluuttaa lohkoketjuun. Sijoitettu kryptovaluutta lukitaan yleensä joksikin ajaksi, jolloin sijoitettuja varoja ei voi käyttää.
\item Palkitseminen, missä osakkaille jaetaan sijoitetun valuutan mukaan suhteutettu palkinto siitä, että he ovat olleet osallisina lohkoketjun turvaamisessa.
\item Rankaiseminen, missä lohkoketjun väärinkäytöstä tai sen yritykseen syyllistyneiden käyttäjien lukitsemista varoista vähennetään jokin osuus.
\item Uudelleenvalinta, missä valitaan satunnaisesti tai muun valintaperusteen nojalla osakkaat, jotka validoivat seuraavan lisäkysen lohkoketjuun.
\end{enumerate}

Proof-of-Stake konsensusmekanismin ollaan havaittu sen eri implementaatioissa olevan hyvin vähän energiaa kuluttava. Ongelmia kuitenkin on: matala yksityisyys, pääoman kasautuminen muutamalle osakkaalle ja skaalautuvuus ovat suurimpia PoS:n kritiikkejä. Näihin on kuitenkin esitetty ratkaisuja esimerkiksi Ethereum 2.0 projektissa ja Cardanossa.

Seuraavaksi tutkielma esittelee kaksi Proof-of-Stake konsensusmekanismilla toimivaa lohkoketjua: Ethereum 2.0:n ja Cardanon. Lohkoketjuja ei ole vielä otettu käyttöön, mutta niiden käyttämistä menetelmistä on useita tutkimuksia.

\subsection{Ethereum 2.0\label{ethereum2.0}}
\begin{otherlanguage}{english}

Ethereum 2.0

\end{otherlanguage}
\subsection{Cardano\label{cardano}}
\begin{otherlanguage}{english}

Cardanon on kehittänyt Charles Hoskinsonin perustama IOHK vuonna 2015 \cite{cardanowhitepaper, iohk}, ja Cardanon sekä sen ekosysteemin kehitystä valvoo itsenäinen Cardano Foundation. 

Cardano käyttää PoS-protokollanaan viiden eri akateemisen instituution yhteistyössä kehittämää Ouroboros-lohkoketjuprotokollaa \cite{cardanowhitepaper}. Ouroboros on PoS-konsensusmekanismia soveltava lohkoketjuprotokolla \cite{cardano-ouroboros}, joka on kehitetty ratkaisuksi Bitcoinin korkeaan energiakulutukseen.

Cardanon tavoitteena on rakentua akateemisesti vertaisarvioitujen tutkimuksien pohjalta, ja sen Ourobos-protokolla mahdollistaa useiden eri ominaisuuksien implementoinnin modulaarisesti \cite{cardanowhitepaper}. Ourobos-protokolla mahdollistaa joustavan kehitysprosessin, jonka ansiosta Cardano lohkoketjuna on skaalattavampi ja tulevaisuudenkestävämpi. Cardanossa on erilaisia satunnaislukugeneraattoreita, mitkä ovat muissa lohkoketjuissa harvinaisia sekä mahdollisuus sivuketjuille (side-chains).

Cardanon käyttämä PoS-konsensusmekanismi on energiakulutukseltaan verrattuna PoW- konsensusmekanismeihin alhainen, sillä PoS ei tarvitse lainkaan laskennalliseen tehokkuuteen tai allokoidun tallennustilan suuruuteen perustuvaa todentamista \cite{cardanowhitepaper}. Cardanon energiakulutuksen on arvioitu 2021 olevan 6 gigawattituntia \cite{cardanoenergy}, missä yhden siirtoon vaadittu energiankulutus on keskimäärin 0.5479 kilowattituntia.

\end{otherlanguage}


\end{otherlanguage}