\begin{subsection}{Bitcoin\label{bitcoin}}
% Oikoluettu: X 

Bitcoin kehitettiin vuonna 2009 nimimerkillä "Satoshi Nakamoto" esiintyneen tuntemattoman tahon toimesta ja siitä julkaistiin sen toimintaperiaatetta \cite{bitcoin1, satoshibitcoin} kuvaava tutkimusartikkeli, jota kryptovaluuttakeskusteluissa kutsutaan \textit{white paperiksi}. Bitcoinin tavoitteina fiat-rahaan (perinteinen raha, kuten eurot ja dollarit) verrattuna on, että sitä voi siirtää ilman rajoituksia, sen siirtomaksut ovat alhaisia ja että jokaisella käyttäjillä on pääsy täydelliseen tilikirjaan joka sisältää kaikki tapahtumat. Näiden lisäksi valuuttaa turvaa konsensusmekanismi ja kryptografia. Kryptovaluuttana Bitcoin on myös luottamukseton (eng. \textit{trustless}), eli valuutta ei vaadi toimiakseen mitään kolmatta osapuolta johon käyttäjien täytyisi luottaa vaan se toimii hajautetusti. Tavoitteet ovat muilta osin Bitcoinissa toteutuneet, mutta siirtomaksut ovat kryptovaluutan hinnan noustua nousseet jo kymmeniin euroihin.

Bitcoinissa on vain yhdenlaisia lohkoja, eli siirtolohkoja (eng. \textit{transaction block}). Valuuttaa siirtäessään käyttäjä joutuu maksamaan siirtomaksun (eng. \textit{transaction fee}) ja siirtotapahtuma lisätään seuraavaan lohkoon, jota ei ole vielä lisätty osaksi lohkoketjua. Kun lohko lisätään osaksi lohkoketjua se varmennetaan, eli louhija ratkaisee lohkoon liittyvän pulman ja saa palkintona sen sisältämissä siirtotapahtumissa maksetut siirtomaksut sekä lohkopalkkion (eng. \textit{block reward}). Lohkopalkkioiden määrä puolittuu noin joka neljäs vuosi, ja Bitcoineja tulee olemaan kokonaisuudessaan 21 miljoonaa kappaletta \cite{satoshibitcoin}.

Bitcoinissa ongelman ratkaisemiseksi ja palkkion saamiseksi tärkein tekijä on käyttäjän louhintaan käyttämän tietokoneen laskennallinen tehokkuus \cite{bitcoin1}. Bitcoinissa palkkion saannin todennäköisyys on suoraan verrannollinen käyttäjän laskennallisen tehon suhteeseen lohkoketjun kokonaislaskentatehosta. Tästä syystä valtaosa louhijoista liittyykin niin kutsuttuihin louhintavarantoihin (eng. \textit{mining pool}). Louhintavarannoissa louhijat yhdistävät laskentatehonsa ja saavat louhintavarannon voittamista palkkioista osuutensa riippuen siitä, kuinka suuri osuus heidän laskentatehostaan on louhintavarannon kokonaislaskentatehosta.

Louhinnassa palkkion voittamisen todennäköisyyden riippuessa täysin laskennallisesta tehosta Bitcoinin louhinta on aiheuttanut kilpailua laskentatehosta käyttäjien välillä. Laitteistokilpailusta ja johtavasta asemastaan suurimpana kryptovaluuttana Bitcoin vaatiikin paljon energiaa: vuonna 2021 Bitcoinin louhinnan energiankulutuksen on arvioitu olevan 200,57 terawattituntia \cite{bitcoinenergy}, ja yksi siirtotapahtuma kuluttaa keskimäärin noin 2006,54 kilowattituntia energiaa. Mikäli Bitcoin olisi valtio, olisi Bitcoin energiankulutukseltaan suurempi kuin Thaimaa. Bitcoinin energiankulutus onkin herättänyt keskustelua lohkoketjujen ympäristöystävällisyydestä ja ollut suurena vaikuttajana siihen, miksi ympäristöystävällisempiä uusia konsensusmekanismeja on kehitetty.

Bitcoinin skaalautuvuus on myös saanut osakseen arvostelua. Bitcoin pystyy käsittelemään tällä hetkellä ainoastaan seitsemän siirtotapahtumaa sekunnissa \cite{bitcoin-tps}, ja laskettu teoreettinen maksimi Bitcoinia skaalatessa on 27 siirtotapahtumaa sekunnissa mikäli yhden lohkon koko pysyisi yhdessä megatavussa. Verrattuna siihen, että PoS-konsensusmekanismilla toimivat lohkoketjut pystyvät käsittelemään jo nyt kymmenkertaisia määriä siirtotapahtumia sekunnissa \cite{algorandtech, cardano-ouroboros} on Bitcoinin käyttämän PoW-konsensusmekanismin kohtaama kritiikki tehottomana ymmärrettävää.

\end{subsection}