\subsection{Bitcoin\label{bitcoin}}
\begin{otherlanguage}{english}

Bitcoin kehitettiin vuonna 2009 tuntemattoman henkilön tai henkilöiden toimesta ja siitä julkaistiin silloin white paper, jossa kuvataan sen toimintaperiaate \cite{bitcoin1, satoshibitcoin}. Bitcoinin vahvuuksia fiat-rahaan (perinteinen raha, kuten eurot ja dollarit) verrattuna on, että sitä voi siirtää ilman rajoituksia, sen siirtomaksut ovat alhaisia ja että jokaisella käyttäjillä on pääsy täydelliseen tilikirjaan joka sisältää kaikki tapahtumat. Näiden lisäksi valuuttaa turvaa konsensusmekanismi ja kryptografia.

Bitcoinissa on kahdenlaisia tapahtumia: siirtotapahtumia ja uuden kolikon luomistapahtumia \cite{bitcoin1}. Siirtotapahtumissa valuuttaa siirtävä käyttäjä joutuu maksamaan siirtomaksun (transaction fee). Louhija ratkaisee tapahtumaan liittyvän pulman, ja saa palkintona sen siirtämiseksi maksetun siirtomaksun. Luomistapahtumissa luodaan uusia Bitcoineja, jolloin luomistapahtuman ratkaiseva louhija saa luodut Bitcoinit palkintona tapahtumaan liittyvän ongelman ratkaisusta. Uusien luotujen Bitcoinien määrä puolittuu noin joka neljäs vuosi, ja Bitcoineja tulee olemaan ainoastaan 21 miljoonaa kappaletta \cite{satoshibitcoin}.

Bitcoinissa ongelman ratkaisemiseksi ja palkkion saamiseksi tärkein tekijä on käyttäjän louhintaan käyttämän tietokoneen laskennallinen tehokkuus \cite{bitcoin1}. Bitcoinissa palkkion voittamisen todennäköisyys on suoraan verrannollinen käyttäjän osuuteen koko Bitcoinin louhintaan käytetystä laskentatehosta.

Louhinnassa palkkion saamisen todennäköisyyden riippuessa täysin laskennallisesta tehosta on Bitcoinin louhinta aiheuttanut sen, että louhijat kilpailevat laskentatehossa toistensa kanssa. Johtuen laitteistokilpailusta ja siitä, että Bitcoin on PoW-lohkoketjuista markkinaosuudeltaan suurin vaatii Bitcoin paljon energiaa: vuonna 2021 Bitcoinin louhinnan energiankulutuksen on arvioitu olevan 200,57 terawattituntia \cite{bitcoinenergy}, ja yksi siirtotapahtuma kuluttaa keskimäärin noin 2006,54 kilowattituntia energiaa. Mikäli Bitcoin olisi valtio, olisi Bitcoin energiankulutukseltaan suurempi kuin Thaimaa. Bitcoinin valtava energiankulutus on nostanut esiin keskustelua lohkoketjujen ympäristöystävällisyydestä ja ollut suurena vaikuttajana siihen, miksi vaihtoehtoisia ympäristöystävällisiä konsensusmekanismeja on kehitetty.

Bitcoinin skaalautuvuus on myös saanut osakseen kritiikkiä. Bitcoin pystyy käsittelemään ainoastaan seitsemän siirtotapahtumaa sekunnissa \cite{bitcoin-tps}, ja laskettu teoreettinen maksimi Bitcoinille on 27 siirtotapahtumaa sekunnissa mikäli yhden lohkon koko pidetään yhdessä megatavussa. Verrattuna siihen, että PoS-konsensusmekanismilla toimivat lohkoketjut pystyvät käsittelemään jo nyt tuhansia siirtotapahtumia sekunnissa \cite{algorandtech, cardano-ouroboros} on Bitcoinin käyttämä PoW-konsensusmekanismi pelkästään näiden lukujen kautta tarkasteltuna huonompi.

\end{otherlanguage}