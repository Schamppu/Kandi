\subsection{Bitcoin\label{bitcoin}}
\begin{otherlanguage}{english}

Bitcoin kehitettiin vuonna 2009 tuntemattoman henkilön tai henkilöiden toimesta ja siitä julkaistiin silloin white paper, jossa kuvataan sen toimintaperiaate \cite{bitcoin1, satoshibitcoin}. Bitcoinin vahvuuksia perinteisiin fiat-rahoihin verrattuna on, että sitä voi siirtää ilman rajoituksia, sen siirtomaksut ovat alhaisia ja että jokaisella käyttäjillä on pääsy täydelliseen tilikirjaan joka sisältää kaikki tapahtumat. Näiden lisäksi valuuttaa turvaa konsensusmekanismi ja kryptografia.

Bitcoinissa on kahdenlaisia tapahtumia: siirtotapahtumia ja uuden kolikon luomistapahtumia \cite{bitcoin1}. Siirtotapahtumissa valuuttaa siirtävä käyttäjä joutuu maksamaan siirtomaksun (transaction fee). Louhija ratkaisee tapahtumaan liittyvän pulman, ja saa palkintona sen siirtämiseksi maksetun siirtomaksun. Luomistapahtumissa luodaan uusia Bitcoineja, jolloin luomistapahtuman ratkaiseva louhija saa luodut Bitcoinit palkintona tapahtumaan liittyvän ongelman ratkaisusta. Uusien luotujen Bitcoinien määrä puolittuu noin joka neljäs vuosi, ja Bitcoineja tulee olemaan ainoastaan 21 miljoonaa kappaletta \cite{satoshibitcoin}.

Bitcoinissa ongelman ratkaisemiseksi ja palkkion saamiseksi tärkein tekijä on käyttäjän louhintaan käyttämän tietokoneen laskennallinen tehokkuus \cite{bitcoin1}. Bitcoinissa palkkion voittamisen todennäköisyys on suoraan verrannollinen käyttäjän osuuteen koko Bitcoinin louhintaan käytetystä laskentatehosta.

Louhinnassa palkkion saamisen todennäköisyyden riippuessa täysin laskennallisesta tehosta on Bitcoinin louhinta aiheuttanut sen, että louhijat kilpailevat laskentatehossa toistensa kanssa. Johtuen tästä ja siitä, että Bitcoin on PoW-lohkoketjuista markkinaosuudeltaan suurin on Bitcoinin energiavaativuus myös suurin: vuonna 2021 Bitcoinin louhinnan energiankulutus on arvioitu olevan 181.07 terawattituntia \cite{bitcoinenergy}. 


\end{otherlanguage}