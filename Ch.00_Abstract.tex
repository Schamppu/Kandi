% \begin{abstract}{finnish}

% Tämä dokumentti on tarkoitettu Helsingin yliopiston tietojenkäsittelytieteen osaston opin\-näyt\-teiden ja harjoitustöiden ulkoasun ohjeeksi ja mallipohjaksi. Ohje soveltuu kanditutkielmiin, ohjelmistotuotantoprojekteihin, seminaareihin ja maisterintutkielmiin. Tämän ohjeen lisäksi on seurattava niitä ohjeita, jotka opastavat valitsemaan kuhunkin osioon tieteellisesti kiinnostavaa, syvällisesti pohdittua sisältöä.


% Työn aihe luokitellaan  
% ACM Computing Classification System (CCS) mukaisesti, 
% ks.\ \url{https://dl.acm.org/ccs}. 
% Käytä muutamaa termipolkua (1--3), jotka alkavat juuritermistä ja joissa polun tarkentuvat luokat erotetaan toisistaan oikealle osoittavalla nuolella.

% \end{abstract}

\begin{otherlanguage}{finnish}
\begin{abstract}
Bitcoinin energiankulutuksen on arvioitu vuonna 2021 olevan 200,57 terawattituntia, mikä on kasvanut vuodesta 2017 599,6\%. Bitcoinin ja muiden kryptovaluuttojen energiankulutuksen jatkuva kasvu on herättänyt keskustelua teknologian ympäristöystävällisyydestä. Tämä tutkielma vertailee Proof-of-Work -konsensusmekanismilla toimivien Bitcoinin ja Ethereumin energiatehokkuutta ja ympäristöongelmia uudemmilla konsensusmekanismeilla toimiviin kryptovaluuttoihin. Uusista konsensusmekanismeista tutkielmassa esitellään Proof-of-Stake ja Proof-of-Space käyttäen esimerkkilohkoketjuina vertailussa Cardanoa, Algorandia ja Chiaa. Tutkielman pyrkimyksenä on vertailun kautta osoittaa, että uudet konsensusmekanismit ovat mahdollisia ratkaisuja Proof-of-Workin aiheuttamiin ympäristöongelmiin ja miten niiden hajautuneisuus ja skaalautuvuus vertautuu Proof-of-Workiin.

Tutkielma havaitsi vertailussa, että uudemmista lohkoketjuista Proof-of-Stake saavuttaa tällä hetkellä parhaan energiatehokkuuden. Proof-of-Space -lohkoketjuista Chia saavuttaa tutkielman vertailussa myös paremman energiatehokkuuden kuin Proof-of-Work. Proof-of-Stake -lohkoketjujen hajautuneisuus saavuttaa tutkielman vertailussa vain hieman alhaisemmat tulokset Proof-of-Workiin verrattuna. Proof-of-Spacesta ei ole olemassa sen hajautuneisuudesta tehtyjä tutkimuksia, joten sen hajautuneisuutta ei voitu tutkielmassa vertailla muihin lohkoketjuihin. 

Elektroniikkajätteen, hiilidioksidipäästöjen ja energiankulutuksen perusteella Proof-of-Work osoittautui tutkielmassa varsin ongelmalliseksi konsensusmekanismiksi, ja uusien konsensusmekanismien nähtiin toimivan ekologisessa kontekstissa huomattavasti paremmin ilman, että niiden hajautuneisuus tai skaalautuvuus olisivat kärsineet merkittävästi.
\end{abstract}
\end{otherlanguage}
