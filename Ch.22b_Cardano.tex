\subsection{Cardano\label{cardano}}
\begin{otherlanguage}{english}

Cardanon on kehittänyt Charles Hoskinsonin perustama IOHK vuonna 2015 \cite{cardanowhitepaper, iohk}, ja Cardanon sekä sen ekosysteemin kehitystä valvoo itsenäinen Cardano Foundation. 

Cardano käyttää PoS-protokollanaan viiden eri akateemisen instituution yhteistyössä kehittämää Ouroboros-lohkoketjuprotokollaa \cite{cardanowhitepaper}. Ouroboros on PoS-konsensusmekanismia soveltava lohkoketjuprotokolla \cite{cardano-ouroboros}, joka on kehitetty ratkaisuksi Bitcoinin korkeaan energiakulutukseen.

Cardanon tavoitteena on rakentua akateemisesti vertaisarvioitujen tutkimuksien pohjalta, ja sen Ourobos-protokolla mahdollistaa useiden eri ominaisuuksien implementoinnin modulaarisesti \cite{cardanowhitepaper}. Ourobos-protokolla mahdollistaa joustavan kehitysprosessin, jonka ansiosta Cardano lohkoketjuna on skaalattavampi ja tulevaisuudenkestävämpi. Cardanossa on erilaisia satunnaislukugeneraattoreita, mitkä ovat muissa lohkoketjuissa harvinaisia sekä mahdollisuus sivuketjuille (side-chains).

Cardanon käyttämä PoS-konsensusmekanismi on energiakulutukseltaan verrattuna PoW- konsensusmekanismeihin alhainen, sillä PoS ei tarvitse lainkaan laskennalliseen tehokkuuteen tai allokoidun tallennustilan suuruuteen perustuvaa todentamista \cite{cardanowhitepaper}. Cardanon energiakulutuksen on arvioitu 2021 olevan 6 gigawattituntia \cite{cardanoenergy}, missä yhden siirtoon vaadittu energiankulutus on keskimäärin 0.5479 kilowattituntia.

\end{otherlanguage}