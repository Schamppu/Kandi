\section{Hiilidioksidipäästöt\label{hiilidioksidipaastot}}
\begin{otherlanguage}{english}

PoW on hiilidioksidipäästöjen osalta konsensusmekanismeista suurin saastuttaja. Energiankulutuksen ja louhintaan erikoistuneiden laitteiden valmistamisen arvioitiin vuonna 2019 aiheuttaneen kokonaisuudessaan 39,27 megatonnia hiilidioksidipäästöjä Bitcoinin osalta \cite{btc-carbon-ewaste}. Suomen laskettiin vuonna 2019 aiheuttaneen 42,55 megatonnia hiilidioksidipäästöjä \cite{carbon-finland}, eli Bitcoin vastaa miltei Suomen vuosittaista hiilijalanjälkeä. Ethereumin vaatiessa vastaavanlaista erikoistunutta laitteistoa ja sen energiankulutuksen ollessa noin puolet siitä miten paljon Bitcoin tällä hetkellä vaatii energiaa vuodessa, voidaan arvioida Ethereumin hiilijalanjäljen olevan noin puolet Bitcoinin hiilijalanjäljestä.

Hiilidioksidipäästöjen laskeminen PoSp:lle on huomattavasti hankalampaa, sillä aihetta ei ole vielä tutkittu. Kovalevyjen valmistamisen arvioitiin Euroopan unionin komission toimesta vaativan noin kaksi kilowattituntia energiaa \cite{manufacturingcarbon1}, mutta on hankalaa arvioida kuinka moni Chian louhijoista ostaa uusia kovalevyjä louhintaan. Energiankulutuksesta voidaan kuitenkin laskea, että kun De Loosen ja kollegoiden tutkimuksessa \cite{btc-carbon-ewaste} esitetyn arvion mukaan kryptolouhinta aiheuttaa keskimäärin \(473.64~g~CO_{2}/kWh \), aiheuttaisi Chia 0,1454 megatonnia hiilidioksidipäästöjä vuodessa.

PoS:n energiankulutuksen ollessa vuosittain vain joitakin gigawattitunteja ja kun PoS ei aiheuta merkittävää uusien laitteiden ostamista ja näin ollen elektroniikkajätettä, sen hiilidioksidipäästöt ovat marginaaliset sekä PoW:n että PoSp:n verrattuna.

\end{otherlanguage}